\chapter*{Abstract}
%Despite the remarkable success of the Standard Model of particle physics in explaining particle collision phenomena, strong efforts were made to develop theories that go beyond the Standard Model.  
%This is mainly caused by ``shortcomings'' of the Standard Model that are considered to cause severe problems.
%For instance, the Standard Model does not contain a suitable Dark Matter candidate and suffer from quadratic divergencies in the calculation of the Higgs boson mass.
%One of the beyond Standard Model theories introduce new fermionic generators which relate bosons with fermions and vice versa.
%This makes the Lagrangian density invariant under a fully new symmetry, a so-called supersymmetry.
%Supersymmetric models can lead to a variety of different phenomenologies.

This thesis presents a search for physics beyond the Standard Model by the selection of highly ionising, short tracks, and a measurement of the jet transverse-momentum resolution using \GAMJET events. 

The search for physics beyond the Standard Model is motivated by supersymmetric models with nearly mass-degenerate lightest neutralinos and next-to lightest charginos.
The small mass gap between chargino and neutralino can lead to long lifetimes of the chargino due to phase space suppression.
Thus, the chargino can reach the detector before its decay.
The here presented search targets chargino lifetimes of $\ctau \approx 1 - 30\cm $ where most of the charginos even decay in the first layers of the tracker. 
This search aims in increasing the search sensitity of exisiting searches with respect to these modesl in a twofold way: First, the inclusion of tracks down to three measurements in the tracking system, and second, the discrimination against Standard Model background by the variable \dedx, the energy loss per path length.
The search is performed on 19.7\fbinv of data recorded at the CMS experiment at a centre-of-mass energy of 8\tev.
The background is mainly estimated using data-based techniques and consists mostly of fake tracks, \ie tracks no associated to one single particle.
The search is performed in four exclusive signal regions in order to enhance the search sensitity with repect to different chargino masses and lifetimes.
No excess above the Standard Model expectation is found and the supersymmetric parameter space is constrained.
The search can exclude supersymmetric models with chargino masses of 100\gev down to lifetimes of $\ctau=2\cm$ and for masses of 500\gev down to lifetimes of $\ctau=70\cm$.
Current limits could be confirmed and improvements of the order of 10-40 GeV were achieved.

In the second part of the thesis, a measurement of the jet transverse-momentum resolution at 8\tev at the CMS experiment is presented.
In order to exploit the good calorimeter energy resolution of the CMS experiment, the measurement is performed using \GAMJET events, where the photon energy can be used as a measure for the true jet transverse momentum. 
The applied method is based on earlier measurements but is further developed within this thesis in order to consistently account for the influence of additional jet activity on the jet transverse-momentum response.
By this development a well behaved method could be ensured.

\newpage 
\chapter*{Zusammenfassung}
In der hier praesentierten Doktorarbeit wird eine Suche nach Physik jenseits des Standardmodels und eine Messung der Jet Transversalimpulsaufloesung vorgestellt.

Die Suche nach Physik jenseits des Standardmodells ist motiviert durch supersymmetrische Modelle, die ein leichtestes Neutralino masseentartet mit dem leichtesten Chargino beinhalten.


\cleardoublepage
