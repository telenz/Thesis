%%%%%%%%%%%%%%%%%%%%%%%%%%%%%%%%%%%%%%%%%%%%%%%%%%%%%%%%%%%%%%%%%%%%%%%%%%%%%%%%%%%%%%%%%%%%%%%%%%%%%%%%%%%%%%%%%%%%%%%%%%%%%%%%%%%%%%%%%%%%%%%%%%%%%%%%%%%%%%%%%%%%%%%%%%%%%%%%%%%%
%%%%%%%%%%%%%%%%%%%%%%%%%%%%%%%%%%%%%%%%%%%%%%%%%%%%%%%%%%%%%%%%%%%%%%%%%%%%%%%%%%%%%%%%%%%%%%%%%%%%%%%%%%%%%%%%%%%%%%%%%%%%%%%%%%%%%%%%%%%%%%%%%%%%%%%%%%%%%%%%%%%%%%%%%%%%%%%%%%%%
\FloatBarrier
\chapter{Motivation}
\label{sec:Motivation}
R-parity conserving supersymmetric models are able to offer solutions to many unexplained phenomena in astrophysics and can solve many of the shortcomings of the Standard Model of particle physics (see Section~\ref{ch:Supersymmetry}).
While supersymmetric models, especially the Minimal Supersymmetric Standard Model (MSSM) (Section~\ref{sec:MSSM}), have been studied at previous particle colliders including Tevatron and LEP~\cite{bib:Tevatron:SUSY_results,bib:LEP:SUSY_results}, the LHC with its high centre-of-mass energy offers a unique opportunity to investigate SUSY models with high particle masses that were not accessible in previous experiments.

Therefore, a variety of searches were hunting for SUSY during Run\,I of the LHC from 2010 to 2012.
Proton-proton collision data from the CMS and ATLAS experiments were analysed with a strong focus on the search for SUSY in production channels via the strong interaction (\eg~\cite{bib:CMS:RA2_8TeV,bib:CMS:MT2_8TeV,bib:ATLAS:JetPlusMET_8TeV}).
As a consequence, wide, previously unexplored regions of the MSSM parameter space are already excluded.
However, due to the unknown mechanism of supersymmetry breaking, the most general parametrisation of the MSSM introduces over 100 new parameters and thus opens up an incredibly large phenomenological space. 
Therefore, SUSY models can lead to a plethora of possible signatures at particle colliders, many of which could not - or not fully - be explored. \\

A very interesting signature occurs when particles live long enough to travel through a part or the whole detector before decaying.
This is possible for SUSY models with compressed spectra, in which a particle can be long-lived because of phase-space suppression.
In the MSSM, such a mass-degeneracy naturally occurs if the wino mass parameter ($M_2$) is smaller than the bino ($M_1$) and higgsino ($\mu$) mass parameters.
In this case, the lightest chargino (\chipm) and the lightest neutralino (\chiO) are both wino-like and their mass gap is fully determined by higher order corrections (see Section~\ref{ch:Longlived_Particles}). 
Therefore, they are almost mass-degenerate and the chargino is long-lived.

Such scenarios can be very interesting from a cosmological perspective as the wino-like lightest supersymmetric particle, the neutralino \chiO, can serve as a plausible Dark Matter candidate~\cite{bib:Ibe:DarkMatter_2015,bib:Hisano:DarkMatter_2005}.
While it is not possible to explain the full relic density with thermally produced wino-like neutralinos for m$_{\chiO}\lesssim 3\tev$~\cite{bib:IndirectSearches_Cohen_2013}, neutralinos can still be the dominant part if they are non-thermally produced via the decay of an almost decoupled particle~\cite{bib:Moroi:DarkMatter_1999,bib:Moroi:DarkMatter_2013}.

Additionally, explorations of the MSSM parameter space reveal that many models that are consistent with current observations and theoretical constraints and that offer a neutralino as Dark Matter candidate include a metastable chargino with a mass gap of the order of~$\sim160\mev$ with respect to the neutralino~\cite{bib:pMSSMScan_2013}.\\

SUSY scenarios with nearly mass-degenerate particles have two distinctive phenomenological properties that require a very different search strategy compared to general SUSY searches. 
First, because of the mass-degeneracy, the remaining decay product (\eg a pion) is very soft in \pt, making it hard to detect.
Since the other decay product, the neutralino, is only weakly interacting, it is very difficult to identify charginos via their decay products. 
Second, as the chargino is long-lived, it may traverse several detector layers before decaying.
Thus, there is the possibility of reconstructing the chargino itself, \eg as a reconstructed track in the tracker system.\\

Despite the exotic signatures of supersymmetric models with nearly mass-degenerate \chipm and \chiO, current CMS searches are already sensitive to a very broad range of lifetimes.
The exclusion power of existing SUSY searches can be assessed by interpreting their results in terms of the fraction of excluded parameter points in the phenomenological MSSM (see Section~\ref{subsec:pMSSM} for an introduction to the pMSSM). 
The results of such a study which has been performed in~\cite{bib:CMS:DT_8TeV} are shown in Figure~\ref{fig:pMSSMplot}. 
\begin{figure}[!b]
\vspace{20pt}
  \centering 
  \begin{tabular}{c}
    \includegraphics[width=0.75\textwidth]{figures/analysis/pMSSM_vs_ctau.pdf}
  \end{tabular}
  \caption{The number of excluded pMSSM points at 95\% C.L. (upper part) and the fraction of excluded pMSSM points (bottom part) vs. the chargino lifetime for different CMS searches.
           Red area: the search for long-lived charged particles \cite{bib:CMS:HSCP_8TeV},
           Purple area: the search for disappearing tracks  \cite{bib:CMS:DT_8TeV},
           Blue area: a collection of various general SUSY searches \cite{bib:CMS:pMSSMinterpretation_7TeV_PAS}
           The black line indicates the unexcluded pMSSM parameter points.
           The sampling of the parameter space points was done according to a prior probability density function which takes pre-LHC data and results from indirect SUSY searches into account (see \cite{bib:CMS:HSCPReinterpreation_PAS} for further details).
           Taken from: \cite{bib:pMSSMplot_source_from_DT}.}
  \label{fig:pMSSMplot}
\vspace{20pt}
\end{figure}
It can be seen that general SUSY searches (blue area) are sensitive to shorter chargino lifetimes ($c\tau \lesssim 1\cm$).\footnote{Since the pMSSM interpretation relied on the use of fast simulation techniques which are not capable of simulating charginos with lifetimes $\ctau>1\cm$, the general SUSY searches were never interpreted in the context of SUSY models with longer chargino lifetimes.} 
Two existing searches, the search for long-lived charged particles~\cite{bib:CMS:HSCP_8TeV} and the search for disappearing tracks~\cite{bib:CMS:DT_8TeV} focus on long and intermediate chargino lifetimes, respectively. 
These two searches (purple and red areas) are sensitive to chargino lifetimes of $\ctau \gtrsim 10\cm$.
Taken together, the existing searches exclude a large fraction of pMSSM points at different chargino lifetimes. 
However, there is a gap between the general SUSY searches and the search for disappearing tracks that is not accessible by any of the existing searches.\\

The here presented analysis aims at targeting this gap by optimising the search strategy for charginos with intermediate lifetimes of $1\cm \lesssim c\tau \lesssim 30\cm$. 
The targeted optimisation strategy is a combination of the strategies used in the search for long-lived charged particles~\cite{bib:CMS:HSCP_8TeV} and the search for disappearing tracks~\cite{bib:CMS:DT_8TeV}.
While in~\cite{bib:CMS:HSCP_8TeV}, the high ionisation loss of hypothetical new massive particles is exploited, the analysis does not take into account whether the associated reconstructed tracks are disappearing.
In~\cite{bib:CMS:DT_8TeV}, the disappearance of the track is utilised but the analysis does not incorporate the large ionisation losses into the search.
Additionally, neither of the two searches take into account the possibly very short tracks of early decaying charginos.\\
\newpage
Thus, the here presented search is the first analysis at CMS combining the two signature properties that are highly distinctive for charginos with intermediate lifetimes: 
first, the characteristically high ionisation losses of heavy charginos;
second, short reconstructed tracks due to chargino decays early in the detector.

The associated challenges and the general search strategy of this analysis will be presented in the next section.

%%%%%%%%%%%%%%%%%%%%%%%%%%%%%%%%%%%%%%%%%%%%%%%%%%%%%%%%%%%%%%%%%%%%%%%%%%%%%%%%%%%%%%%%%%%%%%%%%%%%%%%%%%%%%%%%%%%%%%%%%%%%%%%%%%%%%%%%%%%%%%%%%%%%%%%%%%%%%%%%%%%%%%%
%%%%%%%%%%%%%%%%%%%%%%%%%%%%%%%%%%%%%%%%%%%%%%%%%%%%%%%%%%%%%%%%%%%%%%%%%%%%%%%%%%%%%%%%%%%%%%%%%%%%%%%%%%%%%%%%%%%%%%%%%%%%%%%%%%%%%%%%%%%%%%%%%%%%%%%%%%%%%%%%%%%%%%%
