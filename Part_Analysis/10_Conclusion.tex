%%%%%%%%%%%%%%%%%%%%%%%%%%%%%%%%%%%%%%%%%%%%%%%%%%%%%%%%%%%%%%%%%%%%%%%%%%%%%%%%%%%%%%%%%%%%%%%%%%%%%%%%%%%%%%%%%%%%%%%%%%%%%%%%%%%%%%%%%%%%%%%%%%%%%%%%%%%%%%%%%%%%%%%
%%%%%%%%%%%%%%%%%%%%%%%%%%%%%%%%%%%%%%%%%%%%%%%%%%%%%%%%%%%%%%%%%%%%%%%%%%%%%%%%%%%%%%%%%%%%%%%%%%%%%%%%%%%%%%%%%%%%%%%%%%%%%%%%%%%%%%%%%%%%%%%%%%%%%%%%%%%%%%%%%%%%%%%
\clearpage
\FloatBarrier
\chapter{Discussion and conclusion}
\label{sec:Discussion}

The here presented search for highly ionising, short tracks is motivated by supersymmetric models with almost mass-degenerate wino-like charginos \chipm and neutralinos \chiO.
Such scenarios can have interesting astrophysical impacts~\cite{bib:Moroi:DarkMatter_2013} and occur naturally in Supersymmetry, if the wino mass parameter is smaller than the bino and higgsino mass parameters.\\
%Such scenarios are well motivated by astrophysical observations that suggest the existence of large amounts of dark matter.
%It is possible to explain the relic density with wino-like neutralinos if they are non-thermally produced via the decay of long-lived particles, such as gravitinos (bla)~\cite{bib:Moroi:DarkMatter_2013}.\\

The presented analysis is designed to increase the search sensitivity on SUSY models with low chargino lifetimes.
It extends the search for disappearing tracks~\cite{bib:CMS:DT_8TeV} by the inclusion of the variable \dedx.
In order to increase the search sensitivity with respect to short lifetimes, energy information from the pixel silicon tracker is taken into account.
For this purpose, a dedicated pixel energy calibration was carried out within this thesis to ensure stable energy measurements over time and across pixel modules.
This is thus the first analysis at CMS that makes use of energy information from the pixel tracker.
By adding pixel energy information the discrimination power of \dedx is significantly increased.

Overall, \dedx inclusion allows for loosening the requirement on the number of hits in the tracker with respect to~\cite{bib:CMS:DT_8TeV} that leads to a strong suppression of signal events for low chargino lifetimes. 
The Asymmetric Smirnov discriminator, \ias, which is used for \dedx discrimination in this analysis, shows good separation power and can lead to sensitivity increases up to 400\% in this search (cf. Fig.~\ref{fig:optimisation}).\\

The Standard Model background is mainly estimated with data-based techniques.
The main background to this search is arising from fake tracks, \ie tracks that are reconstructed out of the hits from several particles.
Fake tracks are typically short and can have large values of \ias, thus showing a very signal-like signature in the detector.
The uncertainty on the fake background is dominated by systematic uncertainties originating from low statistical precision in the simulated datasets.
Simulating more events could therefore significantly improve the search sensitivity.
This strategy is however technically challenging, since storage capacity limits were already reached within the current analysis.
Still, reducing the uncertainty will be one of the main tasks in order to increase the search sensitivity.

Even though this search already features low background, a further background suppression is desirable.
However, the impact on the search sensitivity will be limited because of the high relative Poisson error on low background predictions.
For instance, a reduction of the number of background events by 50\% from 2 to 1 event reduces the signal yield required for a 5$\sigma$-discovery by around 8\%, whereas a 50\% reduction of expected background events from 200 to 100 reduces the required signal yield by 26\%.

Thus, in order to achieve an increase in the search sensitivity in a future analysis, the focus should be put on the other front of the search sensitivity: the signal acceptance.
First and foremost, it is important to lower the signal losses due to the trigger requirements (ISR trigger).
For this purpose, a dedicated track trigger is indispensable.
Such an implementation balb ala
For Run\,II, a dedicated track trigger .

Furthermore, an implementation of a dedicated track reconstruction algorithm optimised for short tracks could increase the reconstruction efficiency of possible chargino tracks, which is currently $\sim$40\% for chargino tracks with $3-4$ hits.
Additionally, a track reconstruction optimised for the reconstruction of soft particles that are not produced in the primary vertex could, on the other hand, make it possible to detect the Standard Model decay products of the charginos and lead to a better discrimination against the Standard Model background.\\


In the current analysis, the background is estimated at $19-24$ events in the low \ias signal regions and $2.5-2.6$ events in the high \ias regions.
This background estimate is confronted with collision data recorded during the year 2012 at the CMS experiment at a centre-of-mass energy of 8\tev.
%Collision data recorded during the year 2012 at the CMS experiment at a centre-of-mass energy of 8\tev was analysed with respect to SUSY scenarios with almost mass-degenerate charginos and neutralinos.
No evidence for physics beyond the Standard Model is found. % since the observations are compatible with SM expectations within $1\sigma$.
Thus, the absence of any deviation from the Standard Model prediction is used to constrain the supersymmetric parameter space.
Wino-like charginos are excluded down to lifetimes of $\ctau=2\cm$ for $m_{\chipm}=100\gev$.
For high mass scenarios of $m_{\chipm}=500\gev$, the excluded lifetime ranges between $\ctau=70-500\cm$.
This confirms the parameter exclusion limits of the search for disappearing tracks~\cite{bib:CMS:DT_8TeV}.
Interestingly, the signal regions of the here presented search and the search from~\cite{bib:CMS:DT_8TeV} show little overlap.
Therefore, this analysis serves as an independent cross check of~\cite{bib:CMS:DT_8TeV}.
Improvements of the exclusion of SUSY models with respect to existing searches of around $10-40\gev$ in chargino mass are achieved in the low lifetime region.\\

Still, SUSY models with higher mass and low lifetimes could not be fully excluded by any search at the LHC.
Therefore, it will stay interesting to further search for these phenomenologically interesting scenarios in collisions at a centre-of-mass energy of 13\tev.
In this context, the here presented approach of combining a selection of short tracks with \dedx information is especially promising.
The expected cross section for SUSY models with higher chargino masses of around 500\gev will increase by a factor of two at $\sqrt{s}= 13\tev$.
Since \dedx is much more discriminating for high masses, the sensitivity impact is expected to be even more pronounced for 13\tev data.

%Up to now, no evidence for Supersymmetry at the LHC could be found.
%Still, such an explanatory powerful and aesthetically convincing theory will not be given up without a certain confidence that it is not realised in nature.
%That makes especially exotic searches very interesting and it will remain requisite to look also in the little exotic corners of Supersymmetry in the future.

%\begin{itemize}
%\item Include a selection of not more than seven hits to be able to combine with DT search , similar sensitivity , combination would be promising
%\item Track trigger will be important to have better signal acceptance
%\item Further background suppresion with better track reconstruction, recontruct secondaries.
%\item Prepare for questions about the upgrade
%\end{itemize}


\begin{itemize}
\item Theoretical motivation
\item What have I done (pixel calibration, other search design)
\item How is the SM backgrund estimated
\item What could be improved
\item What are the results
\item What is a future oyutlook
\end{itemize}

Talk to Markus.
