%%%%%%%%%%%%%%%%%%%%%%%%%%%%%%%%%%%%%%%%%%%%%%%%%%%%%%%%%%%%%%%%%%%%%%%%%%%%%%%%%%%%%%%%%%%%%%%%%%%%%%%%%%%%%%%%%%%%%%%%%%%%%%%%%%%%%%%%%%%%%%%%%%%%%%%%%%%%%%%%%%%%%%%
%%%%%%%%%%%%%%%%%%%%%%%%%%%%%%%%%%%%%%%%%%%%%%%%%%%%%%%%%%%%%%%%%%%%%%%%%%%%%%%%%%%%%%%%%%%%%%%%%%%%%%%%%%%%%%%%%%%%%%%%%%%%%%%%%%%%%%%%%%%%%%%%%%%%%%%%%%%%%%%%%%%%%%%

\FloatBarrier
\chapter{Conclusion and outlook}
%\chapter{Discussion and conclusion}
\label{sec:Discussion}

The here presented search for highly ionising, short tracks is motivated by supersymmetric models with almost mass-degenerate wino-like charginos \chipm and neutralinos \chiO.
Such scenarios can have interesting astrophysical implications~\cite{bib:Moroi:DarkMatter_2013} and occur naturally in supersymmetric models, if the wino mass parameter is smaller than the bino and higgsino mass parameters.\\
%Such scenarios are well motivated by astrophysical observations that suggest the existence of large amounts of Dark Matter.

The presented analysis targets SUSY models with intermediate chargino lifetimes. 
This is achieved by searching for isolated, high \pt tracks that are highly ionising.
No requirement on the number of tracker hits is enforced, thus, possibly very short tracks from early decaying charginos are included in this analysis.
It, thus, extends the search for disappearing tracks~\cite{bib:CMS:DT_8TeV} by the inclusion of the variable \dedx and the loosening of the requirement on the number of hits in the tracker ($\nhits\geq3$) that leads to a strong suppression of signal events for low chargino lifetimes (cf. Fig.~\ref{fig:NHits_2Signal_noSelection_normalized}).
It is thereby the first analysis at CMS that studies disappearing tracks with down to three hits.

%Overall, \dedx inclusion allows for loosening the requirement on the number of hits in the tracker with respect to~\cite{bib:CMS:DT_8TeV} that leads to a strong suppression of signal events for low chargino lifetimes. 
In order to increase the search sensitivity with respect to shorter lifetimes, energy information from the pixel silicon tracker is taken into account.
For this purpose, a dedicated pixel energy calibration was carried out within this thesis to ensure stable energy measurements over time and across pixel modules.
This is thus the first time that an analysis at CMS makes use of energy information from the pixel tracker.
By adding pixel energy information, the discrimination power of \dedx is substantially increased (cf. Fig.~\ref{fig:ROCplots}).\\
%The Asymmetric Smirnov discriminator, \ias, which is used for \dedx discrimination in this analysis, shows good separation power and can lead to sensitivity increases up to 400\% in this search (cf. Fig.~\ref{fig:optimisation}).\\

The Standard Model background is mainly estimated with data-based techniques.
The main background to this search is arising from fake tracks, \ie tracks that are reconstructed out of the tracker hits of more than one particle.
Fake tracks are typically short and can have large values of \ias, thus showing a very signal-like signature in the detector.
The background contribution by leptons that are passing the lepton veto is very small and in most of the signal regions almost negligible.

In the current analysis, the background is estimated at 19 and 24 events in the low \ias signal regions and 2.5 and 2.6 events in the high \ias regions.
This background estimate is confronted with collision data recorded during the year 2012 at the CMS experiment at a centre-of-mass energy of 8\tev.
%Collision data recorded during the year 2012 at the CMS experiment at a centre-of-mass energy of 8\tev was analysed with respect to SUSY scenarios with almost mass-degenerate charginos and neutralinos.
No evidence for physics beyond the Standard Model is found. % since the observations are compatible with SM expectations within $1\sigma$.
Thus, the absence of any deviation from the Standard Model prediction is used to constrain the supersymmetric parameter space.
Wino-like charginos are excluded down to lifetimes of $\ctau=2\cm$ for $m_{\chipm}=100\gev$.
For high mass scenarios of $m_{\chipm}=500\gev$, the excluded lifetime ranges between $\ctau=70-500\cm$.
This confirms the parameter exclusion limits of the search for disappearing tracks~\cite{bib:CMS:DT_8TeV}.
Interestingly, the signal regions of the here presented search and the search from~\cite{bib:CMS:DT_8TeV} show a rather small overlap for short chargino lifetimes.
Therefore, this analysis yields a complementary result with respect to the search for disappearing tracks~\cite{bib:CMS:DT_8TeV}.
In summary, the exclusion of SUSY models with respect to earlier searches could be independently confirmed and improvements in the exclusions of around $10-40\gev$ in chargino mass in the low lifetime region are achieved.\\
%Improvements of the exclusion of SUSY models with respect to existing searches of around $10-40\gev$ in chargino mass are achieved in the low lifetime region.\\


While this analysis is able to exclude many SUSY models with intermediate lifetime charginos, there are several promising avenues for even enhancing the search sensitivity.

First, since the sensitivity of the current analysis is mainly limited by large systematic uncertainties originating from low statistical precision in the simulated datasets, simulating more events could significantly improve the search sensitivity.
This strategy is however technically challenging, since storage capacity limits were already reached within the current analysis.
Still, reducing this systematic uncertainty will be one of the main tasks for future research.

Second, even though this search already features low background, a further background suppression is desirable.
However, the impact on the search sensitivity will be limited because of the high relative Poisson error on low background predictions.
For instance - neglecting systematic uncertainties - a reduction of the number of background events by 50\% from 2 to 1 reduces the signal yield required for a 5$\sigma$-discovery by around 8\%, whereas a 50\% reduction of expected background events from 200 to 100 reduces the required signal yield by 26\%.

%Thus, in order to better exploit the potential of the here presented analysis approach, the focus should be on the other determinant of search sensitivity: the signal acceptance.
Thus, in order to improve the here presented analysis, the focus should be on the other determinant of search sensitivity: the signal acceptance.
First and foremost, it is important to lower the signal losses due to trigger requirements.
For this purpose, a dedicated track trigger would be beneficial, especially if a future upgrade would make tracking information available on level one.% and is indispensable and it would be beneficial if future upgrades would make tracking information available on level-one. %introduce a level-one track trigger.

Furthermore, an implementation of a dedicated track reconstruction algorithm optimised for short tracks could increase the reconstruction efficiency of possible chargino tracks, which is currently $\sim$20-40\% for chargino tracks with $3-4$ hits.
Additionally, a track reconstruction optimised for the reconstruction of soft particles that are not produced in the primary vertex could allow for a reconstruction of the Standard Model decay products of charginos, thereby enabling a better discrimination against Standard Model background.\\

%Still, SUSY models with higher mass and low lifetimes could not be fully excluded by any search at the LHC.

In summary, the here presented analysis explored a new path for searching for long-lived charginos decaying early inside the detector.
It is the first analysis that incorporates reconstructed tracks down to three hits.
Furthermore, for the first time, ionisation losses are measured taking energy information from the pixel silicon tracker into account.

As argued, further improvements can allow for accessing new, unexplored SUSY models with long-lived charginos.
Additionally, a search in collisions at a centre-of-mass energy of 13\tev with increased cross sections makes the exploration of SUSY models with higher chargino masses possible.
Since \dedx is much more discriminating for high masses, the inclusion of \dedx in this analysis will become even more powerful. 

%The expected cross section for SUSY models with higher chargino masses of around 500\gev will increase by a factor of two at $\sqrt{s}= 13\tev$.



%In this context, the here presented approach of combining a selection of short tracks with \dedx information is especially promising.
%Up to now, no evidence for Supersymmetry at the LHC could be found.
%Still, such an explanatory powerful and aesthetically convincing theory will not be given up without a certain confidence that it is not realised in nature.
%That makes especially exotic searches very interesting and it will remain requisite to look also in the little exotic corners of Supersymmetry in the future.


