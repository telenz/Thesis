%%%%%%%%%%%%%%%%%%%%%%%%%%%%%%%%%%%%%%%%%%%%%%%%%%%%%%%%%%%%%%%%%%%%%%%%%%%%%%%%%%%%%%%%%%%%%%%%%%%%%%%%%%%%%%%%%%%%%%%%%%%%%%%%%%%%%%%%%%%%%%%%%%%%%%%%%%%%%%%%%%%%%%%%%%%%%%%%%%%%
\FloatBarrier
\chapter{Motivation}
\label{sec:Motivation}
Supersymmetry is able to offer solutions to many unexplained phenomena in astrophysics and can solve many of the shortcomings of the Standard Model of particle physics (see Section~\ref{FIXME}).
While SUSY has been studied at previous particle colliders including Tevatron and LEP~\cite{bib:Tevatron:SUSY_results,bib:LEP:SUSY_results}, the LHC with its high centre-of-mass energy offers a unique opportunity to investigate SUSY models with high sparticle masses that were not accessible in previous experiments.

Therefore, a variety of searches were hunting for SUSY during the run\,I of the LHC in 2011 and 2012.
Proton-proton collision data from the CMS and ATLAS experiments were analysed with a strong focus on the search for SUSY in the strong production sector (\eg~\cite{bib:CMS:RA2_8TeV,bib:CMS:MT2_8TeV,bib:ATLAS:JetPlusMET_8TeV}).
As a consequence, wide, previously unexplored regions of SUSY parameter space are already excluded.
However, due to the unknown mechanism of supersymmetry breaking, the most general parametrisation of the Minimal Supersymmetric Standard Model (MSSM) introduces over 100 new parameters and thus opens up an incredibly large phenomenological space. Therefore, SUSY models can lead to a plethora of possible signatures at particle colliders. \\

(I think it is wrong - What motivation??)
%Among more ``exotic'' SUSY scenarios are models with compressed spectra, where two or more particles are nearly mass-degenerate.
%Especially scenarios with a nearly mass-degenerate lightest chargino (\chipm) and lightest neutralino (\chiO) are very interesting from a theoretical and cosmological perspective as they can help to explain the sources of the relic den%sity~\cite{bib:Moroi:DarkMatter_1999,bib:Hisano:DarkMatter_2005,bib:Ibe:DarkMatter_2015}.
%While it is not possible to explain the full relic density with thermally produced neutralinos for m$_{\chiO}\lesssim 2.9\tev$~\cite{bib:Moroi:DarkMatter_2013}, neutralinos can still be the dominant part if they are non-thermally produ%ced via the decay of a long-lived particle such as a wino-like chargino.
%The enhanced annihilation cross section (called Sommerfeld enhancement) into $WW$- , $ZZ$- or $ff$-pairs for a wino-like dark matter candidate leads to an underprediction of the relic density if the neutralino and chargino masses are t%oo small~\cite{bib:Hisano:DarkMatter_2003}.
%This underprediction can be cured, however, if there is an additional non-thermal production of dark matter that is caused by the decay of a long-lived chargino.
%In Supersymmetry, such a mass-degeneracy naturally occurs in case of wino-like neutralinos and charginos, since the mass gap between $W_{3}$ and $W_{1/2}$ is fully determined by higher loop corrections (see Section~\ref{FIXME}).

%Among more ``exotic'' SUSY scenarios are models with compressed spectra, where two or more particles are nearly mass-degenerate.
%Especially scenarios with a wino-like 
%Especially scenarios with a nearly mass-degenerate lightest chargino (\chipm) and lightest neutralino (\chiO) are very interesting from a theoretical and cosmological perspective as they can help to explain the sources of the relic den%sity~\cite{bib:Moroi:DarkMatter_1999,bib:Hisano:DarkMatter_2005,bib:Ibe:DarkMatter_2015}.
%While it is not possible to explain the full relic density with thermally produced neutralinos for m$_{\chiO}\lesssim 2.9\tev$~\cite{bib:Moroi:DarkMatter_2013}, neutralinos can still be the dominant part if they are non-thermally produ%ced via the decay of a long-lived particle such as a wino-like chargino.
%The enhanced annihilation cross section (called Sommerfeld enhancement) into $WW$- , $ZZ$- or $ff$-pairs for a wino-like dark matter candidate leads to an underprediction of the relic density if the neutralino and chargino masses are t%oo small~\cite{bib:Hisano:DarkMatter_2003}.
%This underprediction can be cured, however, if there is an additional non-thermal production of dark matter that is caused by the decay of a long-lived chargino.
%In Supersymmetry, such a mass-degeneracy naturally occurs in case of wino-like neutralinos and charginos, since the mass gap between $W_{3}$ and $W_{1/2}$ is fully determined by higher loop corrections (see Section~\ref{FIXME}).

Among more ``exotic'' SUSY scenarios are models with compressed spectra, where two or more particles are nearly mass-degenerate.
Especially scenarios with a nearly mass-degenerate lightest chargino (\chipm) and lightest neutralino (\chiO) are very interesting from a theoretical and cosmological perspective, as will be explained immediately.
In R-Parity conserving Supersymmetry, such a mass-degeneracy naturally occurs in case of wino-like neutralinos and charginos, since the mass gap is fully determined by higher loop corrections (see Section~\ref{FIXME}) and thus very small.
A wino-like neutralino on the other hand is attractive from an cosmological perspective, as it can help to explain the sources of the relic density~\cite{bib:Moroi:DarkMatter_1999,bib:Hisano:DarkMatter_2005,bib:Ibe:DarkMatter_2015}.
While it is not possible to explain the full relic density with thermally produced neutralinos for m$_{\chiO}\lesssim 2.9\tev$~\cite{bib:Moroi:DarkMatter_2013}, neutralinos can still be the dominant part if they are non-thermally produced via the decay of a heavy and long-lived particle (moduli field).
The enhanced annihilation cross section (called Sommerfeld enhancement) into $WW$- , $ZZ$- or $ff$-pairs for a wino-like dark matter candidate leads to an underprediction of the relic density if the neutralino mass is too small~\cite{bib:Hisano:DarkMatter_2003}.
This underprediction can be cured, however, if there is an additional non-thermal production of dark matter as explained before.
If the wino mass paramater is much smaller than the bino and higgsino mass parameters, both, the lightest neutralino and the lightest chargino are wino-like and almost mass-degenerate.
Apart from heaving a viable Dark Matter candidate, it would also lead to the prediction of a charged sparticle, that could be detected in the CMS experiment.



SUSY scenarios with nearly mass-degenerate particles have two distinctive phenomenological properties that require a very different search strategy compared to general SUSY searches. 
First, if the chargino and the neutralino are almost mass-degenerate ($\Delta m \lesssim 200\mev$), the remaining decay product (\eg a pion) is very soft in \pt, making it hard to detect.
Second, the chargino is long-lived due to phase space suppression (see Section~\ref{sec:FIXME.Theory:Lifetimes}) and may traverse several detector layers before decaying. \\
%This allows to exploit track charateristics of the chargino like the high ionisation losses due to its high mass.

Although supersymmetric models with nearly mass-degenerate \chipm and \chiO lead to exotic signatures with long-lived charginos and soft decay products, existing SUSY searches at CMS can in principle be sensitive to these models. 
The exclusion power of existing SUSY searches can be assessed by interpreting their results in terms of the fraction of excluded parameter points in the phenomenological MSSM (see Section~\ref{theorySUSY} for an introduction to the pMSSM). 
The results of such a study which has been performed in \cite{bib:CMS:DT_8TeV} are shown in Figure~\ref{fig:pMSSMplot}. 
\begin{figure}[!t]
  \centering 
  \begin{tabular}{c}
    \includegraphics[width=0.75\textwidth]{figures/analysis/pMSSM_vs_ctau.pdf}
  \end{tabular}
  \caption{The number of excluded pMSSM points at 95\% C.L. (upper part) and the fraction of excluded pMSSM points (bottom part) vs. the chargino lifetime for different CMS searches.
           Red area: the search for long-lived charged particles \cite{bib:CMS:HSCP_8TeV},
           Purple area: the search for disappearing tracks  \cite{bib:CMS:DT_8TeV},
           Blue area: a collection of various general SUSY searches \cite{bib:CMS:pMSSMinterpretation_7TeV_PAS}
           The black line indicates the unexcluded pMSSM parameter points.
           The sampling of the parameter space points was done according to a prior probability density function which takes pre-LHC data and results from indirect SUSY searches into account (see \cite{bib:CMS:HSCPReinterpreation_PAS} for further details).
           Taken from: \cite{bib:pMSSMplot_source_from_DT}.}
  \label{fig:pMSSMplot}
\end{figure}
It can be seen that general SUSY searches (blue area) are sensitive to shorter chargino lifetimes ($c\tau \lesssim 10\cm$). 
Due to technical reasons\footnote{The pMSSM interpretation relied on the use of fast simulation techniques which are not capable of simulating charginos with lifetimes $\ctau>1\cm$.}, the general SUSY searches were never interpreted in the context of SUSY models with longer chargino lifetimes. 
Two existing searches, the search for long-lived charged particles~\cite{bib:CMS:HSCP_8TeV} and the search for disappearing tracks~\cite{bib:CMS:DT_8TeV} focus on long and intermediate chargino lifetimes, respectively. 
These two searches (purple and red areas) are sensitive to chargino lifetimes of $\ctau \gtrsim 35\cm$.
Taken together, the existing searches exclude a large fraction of pMSSM points at different chargino lifetimes. 
However, there is a gap between the general SUSY searches and the search for disappearing tracks that is not accessible by any of the existing searches.\\

The here presented analysis aims at targeting this gap by optimising the search strategy for charginos with intermediate lifetimes of $10\cm \lesssim c\tau \lesssim 40\cm$. 
The targeted optimisation strategy is a combination of the strategies used in the search for long-lived charged particles~\cite{bib:CMS:HSCP_8TeV} and the search for disappearing tracks~\cite{bib:CMS:DT_8TeV}.
While in~\cite{bib:CMS:HSCP_8TeV}, the high ionisation losses of hypothetical new massive particles is exploited, it does not take into account whether its reconstructed track is disappearing.
In~\cite{bib:CMS:DT_8TeV}, the disappearance of the track is utilised but it does not incorporate the large ionisation losses into the search.
Additionally, neither of the search does take into account the possible very short tracks of a early decaying chargino.
Thus, the here presented search is the first analysis at CMS combining the two signature properties that are highly distinctive for charginos with intermediate lifetimes: 
first, the characteristically high ionisation losses of heavy charginos;
second, short reconstructed tracks due to chargino decays early in the detector. 

The associated challenges and the general search strategy of this analysis will be presented in the next section.

%%%%%%%%%%%%%%%%%%%%%%%%%%%%%%%%%%%%%%%%%%%%%%%%%%%%%%%%%%%%%%%%%%%%%%%%%%%%%%%%%%%%%%%%%%%%%%%%%%%%%%%%%%%%%%%%%%%%%%%%%%%%%%%%%%%%%%%%%%%%%%%%%%%%%%%%%%%%%%%%%%%%%%%%%%%%%%%%%%%%
%%%%%%%%%%%%%%%%%%%%%%%%%%%%%%%%%%%%%%%%%%%%%%%%%%%%%%%%%%%%%%%%%%%%%%%%%%%%%%%%%%%%%%%%%%%%%%%%%%%%%%%%%%%%%%%%%%%%%%%%%%%%%%%%%%%%%%%%%%%%%%%%%%%%%%%%%%%%%%%%%%%%%%%%%%%%%%%%%%%%
