In the year 2012, 19.7 \fbinv of proton-proton collisions at an unprecedented centre-of-mass energy of 8\tev were recorded at the CMS experiment.
This data has been analysed with a special focus on searches for physics beyond the Standard Model and measurements of Standard Model parameters as well as important performance parameters of the CMS detector.
This thesis contributes in a twofold way to the physics program of CMS.
First, a search for long-lived supersymmetric particles that were not investigated by previous analyses is performed.
Second, the jet transverse-momentum resolution is measured in \GAMJET events at 8\tev.
The following paragraphs summarise the two presented analyses.
A more detailed discussion about the most important findings and an outlook for further research can be found in the last chapters of the corresponding parts, Chapter~\ref{sec:Discussion} and Chapter~\ref{res:ch:Discussion}, respectively.\\

The search for physics beyond the Standard Model that is performed within this thesis is strongly motivated by supersymmetric extensions of the Standard Model that include long-lived charginos decaying inside the tracker into the lightest supersymmetric particle, the neutralino.
Because of the higher masses of supersymmetric particles, the chargino is expected to deposit much higher amounts of energy in the tracking system compared to the Standard Model background.
Thus, a selection on the energy deposit per path length, \dedx, can be highly discriminating.
Furthermore, the search is designed to target supersymmetric models not yet excluded, \ie models with chargino lifetimes of the order of $\ctau \approx 1- 30\cm$ where most of the charginos decay in the first layers of the tracker.
Therefore, the here presented search focuses on the selection of highly ionising, short tracks.
It is the first analysis at CMS that incorporates reconstructed tracks down to three hits and includes \dedx measurements from the innermost detector part, the silicon pixel tracker.
For this purpose, a pixel tracker energy calibration was carried out that assures a uniform energy response across pixel modules and over time.
By the inclusion of pixel energy information, the background suppression for a given signal efficiency is significantly increased by up to one order of magnitude (cf. Fig~\ref{fig:ROCplots}).

The background expectation is mainly estimated with data-based techniques and consists mostly of fake tracks, \ie tracks that are not associated to one single particle. 
Fake tracks can easily mimic the signal signature because of their typically large \dedx values.
A selection is performed in four different signal regions in order to increase the search sensitivity to different chargino lifetimes and masses.
The background estimation of this search is particularly challenging because of the low event yield in most of the control regions.
Therefore, the search sensitivity is mainly limited by systematic uncertainties arising from the limited size of simulated samples and control regions in data.

The search for highly ionising, short tracks is performed on 19.7\fbinv of 8\tev data recorded at the CMS experiment in the year 2012.
The observations are compatible with Standard Model expectations.
This result is used to constrain the supersymmetric parameter space with wino-like charginos.
With this search, it is possible to exclude supersymmetric models with long-lived charginos down to lifetimes of 2\cm for chargino masses of 100\gev and down to 70\cm for masses of 500\gev at a 95\% confidence level.
Current limits are confirmed and improvements of the order of 10-40\gev in chargino mass are achieved.\\

The second contribution of this thesis consists of the measurement of the jet transverse-momentum resolution with \GAMJET events at 8\tev at the CMS detector.
The jet \pt resolution is a crucial ingredient for analyses at CMS relying on a good understanding of the quality of the jet \pt measurement, \eg searches for physics beyond the Standard Model where QCD-multijet background plays a major role~\cite{bib:CMS:RA2_8TeV,bib:CMS:MT2_8TeV,bib:CMS:AlphaT_8TeV} and Standard Model measurements such as the differential $t\overline{t}$ production cross-section~\cite{bib:CMS:TopCrossSection_8TeV}.

The analysis exploits the transverse momentum balance of the photon and the jet in the absence of further jet activity.
Due to the high calorimeter energy resolution, the photon \pt can thus be used as a precise measure of the true jet \pt.
The method of the resolution measurement that is used within this thesis is based on earlier analyses~\cite{bib:CMS:JERCPaper_2011,CMS:PAS:JETResolution_7TeV} but is the first measurement that accounts for the fundamental non-Gaussian behaviour of the measured resolution in exclusive bins of further jet activity.
It is shown, that the Gaussian behaviour can be recovered when separating events by the direction of further jets in the event.
By this improvement, a consistent and well performing method for the jet \pt resolution measurement is established.

Since the main application of the resolution measurement is the adjustment of the simulated resolution to the resolution measured in data, the results are presented as data-to-simulation scale factors.
In four different pseudorapidity regions up to $|\eta|=2.3$, the scale factors vary between 7\% and 20\% with uncertainties between 3\% to 8\%.
The main uncertainty is the uncertainty on the simulation of out-of-cone showering.
The results are in agreement with the jet \pt resolution measurement performed on dijet events~\cite{bib:Kristin_Thesis}.\\
%A puplication of the here presented results of the jet \pt resolution is currently in preparation~\cite{bib:CMS:JERCPaper_InPreparation}.\\

To conclude, this thesis contributes in several ways to the investigation of physics beyond the Standard Model and to a better understanding of the CMS detector performance. %to the investigation of physics beyond the Standard Model and to a better understanding of the performance of the CMS detector.
% high energy particle physics.
First, the search for highly ionising, short tracks helps to gain further insights into supersymmetric models with long-lived charginos and further constrains the supersymmetric parameter space.
Second, the pixel energy calibration significantly improves the discrimination power of the variable \dedx and allows for \dedx measurements of very short tracks.
Finally, an improved method is used to  measure an important performance parameter of the CMS experiment, the jet transverse-momentum resolution.

