In the year 2012, a variety of different searches and measurements were performed at the CMS experiment.
A strong focus was set on the search for physics beyond the Standard Model as well as the measurement of Standard Model parameters and important performance parameters of the CMS detector.
This thesis contributed in a twofold way to the physics program of CMS.

First, a search for physics beyond the Standard Model by the selection of highly ionising, short tracks was performed.
The design of the search was strongly motivated by supersymmetric extensions of the Standard Model that include long-lived charginos decaying inside the tracker into the lightest supersymmetric particle, the neutralino.
Because of the higher masses of supersymmetric particles, the chargino is expected to deposit much higher amounts of energies in the tracking system compared to the Standard Model background.
Additionally, the search targeted supersymmetric models with chargino lifetimes of the order of $\ctau \approx 1- 30\cm$ where most of the charginos even decay in the first layers of the tracker.
Therefore, for the first time, reconstructed tracks down to three tracker hits were incorporated and an energy measurement was performed.
For this purpose, energy information from the silicon pixel detector was exploited for the first time at CMS.
This could only be done by an energy calibration of the pixel tracker which was performed within this thesis.

This search could exclude supersymmetric models with long-lived charginos down to lifetimes of $xxx$ for a mass of 100\gev and down to $xxx$ for 500\gev charginos.
Current limits could be confirmed and slight improvements of the order of 10-30\gev were achieved.
The major challenge of this search consisted in the estimation of the Standard Model background because of the low event yield in most of the control regions.
Therefore, the search sensitivity is mainly limited by systematic uncertainties arising from limited size of simulated samples as well as control regions in data.


The second contribution of this thesis contains the measurment of the jet transverse-momentum resolution at 8\tev at the CMS detector.
The jet \pt resolution is a crucial ingredient for analyses at CMS relying on a good understanding of the quality of the jet \pt measurement, \eg physics beyond the Standard Model where QCD-multijet background plays a major role or Standard Model measurements of the QCD cross section or top quark differental measurements.
The method of the resolution measurement based on earlier methods but is the first measurement that accounted for the fundamental non-Gaussian behaviour of the measured resolution in exclusive bins of further jet activity.
The gaussian behaviour can be recovered when dividing the events by the direction of further jets in the event.

Impressive achievemnts could be achieved.

\begin{itemize}
\item Why are the main achievemnts in both of the analyses
\item What are the main results in both of the analyses
\item Paper in preparation for the resolution measurement
\item Schoener Abschlusssatz
\end{itemize}
