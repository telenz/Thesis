In the year 2012, a variety of different searches and measurements were performed at the CMS experiment.
The main focus was to search for physics beyond the Standard Model as well as to measure Standard Model parameters and important performance parameters of the CMS detector.
This thesis contributed in a twofold way to the physics program of CMS.
First, a search for supersymmetric particles by the selection of highly ionising, short tracks was performed and second the jet transverse-momentum resolution was measured in \GAMJET events at 8\tev.
The following paragraphs summarise the two presented analyses.
A more detailed discussion about the most important findings and possible improvements can be found in the last chapters of the corresponding parts, Chapter~\ref{sec:Discussion} and Chapter~\ref{res:ch:Discussion}, respectively.\\

%The search for physics beyond the Standard Model by the selection of highly ionising, short tracks was performed on 19.7\fbinv of 8\tev data recorded at the CMS experiment in the year 2012.
%The design of the search was strongly motivated by supersymmetric extensions of the Standard Model that include long-lived charginos decaying inside the tracker into the lightest supersymmetric particle, the neutralino.
The search for physics beyond the Standard Model by the selection of highly ionising, short tracks is strongly motivated by supersymmetric extensions of the Standard Model that include long-lived charginos decaying inside the tracker into the lightest supersymmetric particle, the neutralino.
Because of the higher masses of supersymmetric particles, the chargino is expected to deposit much higher amounts of energies in the tracking system compared to the Standard Model background.
Thus, the inclusion of the variable \dedx can be highly discriminating.
Furthermore, the search was designed to target supersymmetric models not yet excluded, \ie models with chargino lifetimes of the order of $\ctau \approx 1- 30\cm$ where most of the charginos even decay in the first layers of the tracker.
Therefore, for the first time, reconstructed tracks down to three tracker hits were incorporated and an energy measurement was performed.
For this purpose, energy information from the silicon pixel detector was exploited for the first time at CMS.
The energy information provided by the pixel tracker could only be used because a well calibrated tracking system was ensured.
This was achieved by the calibration of the pixel tracker energy information which was performed within this thesis.
The inclusion of pixel energy information could increase the background suppression for a given signal efficiency significantly up to one order of magnitude (cf. Fig~\ref{fig:ROCplots}).

The background expectation was estimated mainly with data-based techniques and consisted mostly in fake tracks, \ie tracks that are not associated to one single particle. 
Fake tracks can easily mimic the signal signature because of their typically large \dedx values.
A selection was performed in four different signal regions in order to increase the search sensitivity to different chargino lifetimes and masses.
The major challenges of this search consisted in the estimation of the Standard Model background because of the low event yield in most of the control regions.
Therefore, the search sensitivity is mainly limited by systematic uncertainties arising from the limited size of simulated samples as well as control regions in data.

The search for highly, ionising short tracks was performed with 19.7\fbinv of 8\tev data recorded at the CMS experiment in the year 2012.
The results were compatible with Standard Model expectations.
Thus, this result was used to constrain the supersymmetric parameter space with wino-like charginos.
With this search, supersymmetric models with long-lived charginos down to lifetimes of 2\cm for chargino masses of 100\gev and down to 70\cm for masses of 500\gev could be excluded.
Current limits could be confirmed and improvements of the order of 10-40\gev were achieved.\\


The second contribution of this thesis consists in the measurement of the jet transverse-momentum resolution with \GAMJET events at 8\tev at the CMS detector.
It exploits the transverse momentum balance of the photon and the jet in the absence of further jet activity.
In the method presented in this thesis, the photon \pt can be used as a measure of the true jet \pt in the limit of no further radiation in the event.
Due to the high calorimeter energy resolution, it is therefore able to yield precise results of the true jet \pt.

The jet \pt resolution is a crucial ingredient for analyses at CMS relying on a good understanding of the quality of the jet \pt measurement, \eg physics beyond the Standard Model where QCD-multijet background plays a major role~\cite{bib:CMS:RA2_8TeV,bib:CMS:MT2_8TeV,bib:CMS:AlphaT_8TeV} or Standard Model measurements of the QCD cross section~\cite{bib:CMS:QCD_measurements} or top quark differential measurements~\cite{bib:CMS:TopCrossSection_8TeV}.
The method of the resolution measurement is based on earlier methods but is the first measurement that accounted for the fundamental non-Gaussian behaviour of the measured resolution in exclusive bins of further jet activity.
The Gaussian behaviour can be recovered when separating events by the direction of further jets in the event.
By this separation the validity of the method could be retained and a well performing method was achieved.

Since the main application of the resolution measurement is the adjustment of the simulated resolution to the resolution measured in data, the results are presented as data-to-simulation scale factors.
In various pseudorapidity regions up to $|\eta|=2.3$, the scale factors vary between 7\% and 20\% with uncertainties between 3\% to 8\%, where the main contribution arises due to the uncertainty of the simulation of out-of-cone showering.
They are in agreement with the jet \pt resolution measurement performed on dijet events~\cite{bib:Kristin_Thesis}.\\
%A puplication of the here presented results of the jet \pt resolution is currently in preparation~\cite{bib:CMS:JERCPaper_InPreparation}.\\

To conclude, this thesis contributes in a variety of fields to a better understanding of high energy particle physics.
On the one hand, it helps to gain further understanding of supersymmetric models by the analysis of highly ionising, short tracks which includes the interpretation of the results by constraining the supersymmetric parameter space.
Furthermore, a method of calibrating the energy measurements of the silicon pixel detector was established which ensured an increased discrimination power of the variable \dedx as well as the possibility of including short tracks in the selection.
%Furthermore, by the energy calibration of the silicon pixel detector, necessary for the analysis of short tracks, a method could be established that will even gain importance for future LHC runs because of the accessibility of higher particle masses.
And finally, a measurement of the jet transverse-momentum resolution, which is a very important performance parameter of the CMS experiment, was performed and the used methodology was substantially developed, providing a method which consistently deals with the direction of further jets in the event.

%provides a variety of different methods that are useful for the analysis of collision data at high energies.
%It contributes to a better understanding of supersymmetric models as well as a better understanding of the CMS detector.
%The latter was achieved by the calibration of the silicon pixel tracker which is relevant for an increased sensitivity of the variable \dedx, especially for short tracks, as well as a measurement of the jet transverse-momentum resolution which is a very important performance parameter of the CMS detector.

%To conclude, this thesis provides a variety of different methods that are useful for the analysis of collision data at high energies.
%It provides insights in physics objects - very short reconstructed tracks - that were not used at CMS so far.
%Furthermore, for the first time, ionisation losses are measured taking energy information from the pixel silicon tracker into account.
%Since the use of \dedx will gain more and more importance in the search for long-lived particles due to the exploration of much higher energies, this thesis can contribute to  an sensitivty increase by exploiting pixel information, too.
%Furthermore, the use of \dedx as disciminating variable will gain more and more importance in the search for long-lived particles because of the exploration of much higher energies.
%Furthermore, a energy calibration of the pixel tracker was performed which could especialy be interesting beacuse of \dedx. 
%In the second part of the thesis, the further development of the methodology of an important measurement, namly the jet \pt resolution measurement, ensured that the method can still be used.
%It discussed a precise measurement of an important variable of detector preformance.
 %%%%%%%%%%%%%%%%%%%%%%%%%%%%%%%%%%%%%%%%%%%%%%%%%%%%%%%%%%%%%%%%%%%%%%%%%%%%%%%%%%%%%%%%%%%%
%Vorschlag fuer den letzten Abschnitt:
%Herausheben, dass du sowohl zu einem besseren Verstaendins von SUSY beigetragen hast (Analyse mit limits), inclusive eine Methode etabliert hast, die in Zukunft (13TeV, hoehere Massen) noch wertvoller wird, als auch zu einem besseren Verstaendins des Experiments beigetragen hast (pixel calibration, JER).
%contributed to a better understanding  supersymmerty and better understanding CMS detector (jer, pixel).




