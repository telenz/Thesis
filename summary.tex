In the year 2012, a variety of different searches and measurements were performed at the CMS experiment.
The main tasks were to search for physics beyond the Standard Model as well as to measure Standard Model parameters and important performance parameters of the CMS detector.
This thesis contributed in a twofold way to the physics program of CMS.

First, a search for physics beyond the Standard Model by the selection of highly ionising, short tracks was performed.
The design of the search was strongly motivated by supersymmetric extensions of the Standard Model that include long-lived charginos decaying inside the tracker into the lightest supersymmetric particle, the neutralino.
Because of the higher masses of supersymmetric particles, the chargino is expected to deposit much higher amounts of energies in the tracking system compared to the Standard Model background.
Furthermore, the search was designed to target supersymmetric models not yet excluded, \ie models with chargino lifetimes of the order of $\ctau \approx 1- 30\cm$ where most of the charginos even decay in the first layers of the tracker.
Therefore, for the first time, reconstructed tracks down to three tracker hits were incorporated and an energy measurement was performed.
For this purpose, energy information from the silicon pixel detector was exploited for the first time at CMS.
The energy information provided by the pixel tracker could only be used because a well calibrated tracking system was ensured.
This was achieved by the calibration of the pixel tracker energy information which was performed within this thesis.

The background expectation was estimated mainly with data-based techniques and consisted mostly in fake tracks, \ie tracks that are not associated to one single particle. 
A selection was performed in four different signal regions in order to increase the search sensitivity to different chargino lifetimes and masses.
The major challenges of this search consisted in the estimation of the Standard Model background because of the low event yield in most of the control regions.
Therefore, the search sensitivity is mainly limited by systematic uncertainties arising from limited size of simulated samples as well as control regions in data.

The search for highly, ionising short tracks was performed with 19.7\fbinv of 8\tev data recorded at the CMS experiment in the year 2012.
The results were compatible with Standard Model expectations.
Thus, this result was used to constrain the supersymmetric parameter space with wino-like charginos.
With this search supersymmetric models with long-lived charginos down to lifetimes of 2\cm for chargino masses of 100\gev and down to 70\cm for masses of 500\gev could be excluded.
Current limits could be confirmed and slight improvements of the order of 10-30\gev were achieved.\\


The second contribution of this thesis contains the measurement of the jet transverse-momentum resolution at 8\tev at the CMS detector.
The jet \pt resolution is a crucial ingredient for analyses at CMS relying on a good understanding of the quality of the jet \pt measurement, \eg physics beyond the Standard Model where QCD-multijet background plays a major role or Standard Model measurements of the QCD cross section or top quark differential measurements.
The method of the resolution measurement based on earlier methods but is the first measurement that accounted for the fundamental non-Gaussian behaviour of the measured resolution in exclusive bins of further jet activity.
The Gaussian behaviour can be recovered when separating events by the direction of further jets in the event.
By this separation the validity of the method could be retained and well performing method was achieved.
The results are compatible with the jet \pt measurement performed on dijet events.
As the main application of the resolution measurement is the adjustment of the simulated resolution to the resolution measured in data, the results are presented as data-to-simulation scale factors.
In various pseudorapidity regions up to $|eta|=2.3$, the scale factors vary between 7\% and 20\% with uncertainties between 3\% to 8\%.
The results of this measurement will be accessible in a paper that is still in preparation.\\

With the two presented analyses, this thesis could contribute highly to ????
 





