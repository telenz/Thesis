%%%%%%%%%%%%%%%%%%%%%%%%%%%%%%%%%%%%%%%%%%%%%%%%%%%%%%%%%%%%%%%%%%%%%%%%%%%%%%%%%%%%%%%%%%%%%%%%%%%%%%%%%%%%%%%%%%%%%%%%%%%%%%%%%%%%%%%%%%%%%%%%%%%%%%%%%%%%%%%%%%%%%%%%%%%%%%%%%%%%
%%%%%%%%%%%%%%%%%%%%%%%%%%%%%%%%%%%%%%%%%%%%%%%%%%%%%%%%%%%%%%%%%%%%%%%%%%%%%%%%%%%%%%%%%%%%%%%%%%%%%%%%%%%%%%%%%%%%%%%%%%%%%%%%%%%%%%%%%%%%%%%%%%%%%%%%%%%%%%%%%%%%%%%%%%%%%%%%%%%%
\section{Simulated samples}
\label{sec:SimulatedSamples}

For the search for chargino-like tracks including the investigation of the various backgrounds to this search, this analysis relies also on simulated sample.
An extensive introduction to the techniques and tools required for the simulation of SM and beyond SM processes can be found in Section~\ref{blabla}

In the following two sections an overview about the used SM (Section~\ref{sec:SMSamples}) and SUSY samples (Section~\ref{sec:SignalSamples}) is given.
All samples are reweighted to match the measured distribution of primary vertices in data.

%\begin{itemize}
%\item Event weights applied for correct Pileup modelling
%\item Detailed information can be found in Experimental setup chapter
%\item Think what to write in the detector chapter to be not wiederholend     
%\end{itemize}

\subsection{SM Background samples}
\label{sec:SMSamples}
To investigate the sources of background, various simulated SM samples were used.
Because of the special format required for the investigation of $dE/dx$ variables, not all SM processes were available.
As this analysis needs to rely anyways on a data-based background estimation method, 
this does not constitute a serious problem, but only limit the possibility of an extensive comparison between data and simulation going beyond shape comparisons.

In Table \ref{tab:SMsamples} all SM samples which were available and used in the analysis are listed.


\begin{table}[!t]
\centering
\caption{Standard Model background samples which were used in the analysis.}
\label{tab:SMsamples}
 \makebox[\linewidth]{
\begin{tabular}{llll}
\multicolumn{4}{c}{} \\
\toprule
 Process & Cross section $\left[\pb\right]$ & $\mathcal{O}_{\text{calculation}}$ & Size $\left[\text{TB}\right]$\\
\midrule
 $W$ + jets                                &  36703.2   &  NNLO \cite{bib:FEWZ} & 70.4 \\
&&&\\
 $t\bar{t}$ + jets                         &  245.8    &  NNLO \cite{bib:ttbar:Czakon_2013}& 55.9 \\
&&&\\
 Z$\rightarrow\ell\ell$ ($\ell=e,\mu,\tau$) &  3531.9    &  NNLO \cite{bib:FEWZ} & 5.1  \\
&&&\\
QCD ($50\gev<\pt<1400\gev$)                &  9374794.2 &  LO   & 44.3\\
\bottomrule
\end{tabular}}
\end{table}  


\hspace{5cm}

\renewcommand{\arraystretch}{1.5}
\setlength{\arrayrulewidth}{1pt}
\begin{table}[!b]
\centering
\caption{Standard Model background samples which were used in the analysis.}
\label{tab:SMsamples}
\makebox[\linewidth]{
\begin{tabular}{llll}
\multicolumn{4}{c}{} \\
%\toprule
 Process & Cross section $\left[\pb\right]$ & $\mathcal{O}_{\text{calculation}}$ & Size $\left[\text{TB}\right]$\\

\hline\hline

%\midrule
%\midrule
 $W$ + jets                                &  36703.2   &  NNLO \cite{bib:FEWZ} & 70.4 \\
 $t\bar{t}$ + jets                         &  245.8    &  NNLO \cite{bib:ttbar:Czakon_2013}& 55.9 \\
 Z$\rightarrow\ell\ell$ ($\ell=e,\mu,\tau$) &  3531.9    &  NNLO \cite{bib:FEWZ} & 5.1  \\
QCD ($50\gev<\pt<1400\gev$)                &  9374794.2 &  LO   & 44.3\\
%\bottomrule
\end{tabular}}
\end{table}  
Due to the immense size of the samples (between 10 and 80\,TB), a skimming had to be done to make them handable.
For this a preselection of blabla was applied.

\begin{itemize}
\item Title: Background samples
\item To study the background, following samples were used: ... generated with ...
\item Not all SM processes samples were availbale
\item Not so dramatic, because data based bkg estimation method
\item Main Background are ... ???
\item Only a few were available and thus processed
\item Most important sample was included: Wjets
\item ZoNuNu Backgrund not availbale plays also a role (see DT paper!)
\item Table of SM samples with cross-sections
\end{itemize}

\subsection{Signal samples}
\label{sec:SignalSamples}
\begin{itemize}
\item Lifetime reweighting
\item Simulation of lifetime in Geant
\item Trigger emulation
\item What samples are exactly generated.
\item Madgraph+Pythia
\item Chargino Pair production + Chargino neutralino production
\item Table with generated signal samples with cross-sections???? No
\end{itemize}

%%%%%%%%%%%%%%%%%%%%%%%%%%%%%%%%%%%%%%%%%%%%%%%%%%%%%%%%%%%%%%%%%%%%%%%%%%%%%%%%%%%%%%%%%%%%%%%%%%%%%%%%%%%%%%%%%%%%%%%%%%%%%%%%%%%%%%%%%%%%%%%%%%%%%%%%%%%%%%%%%%%%%%%%%%%%%%%%%%%%
%%%%%%%%%%%%%%%%%%%%%%%%%%%%%%%%%%%%%%%%%%%%%%%%%%%%%%%%%%%%%%%%%%%%%%%%%%%%%%%%%%%%%%%%%%%%%%%%%%%%%%%%%%%%%%%%%%%%%%%%%%%%%%%%%%%%%%%%%%%%%%%%%%%%%%%%%%%%%%%%%%%%%%%%%%%%%%%%%%%%
\section{Event selection}
\label{sec:EventSelection}
\subsection{Datasets and triggers}
\begin{itemize}
\item Datasets and triggers used in the analysis
\item signal samples generated with Madgraph and pythia
\item They are decayed in Geant to only pions. Around ten different lifetimes were simulated
\item For other lifetimes: lifetime reweighting is done PLOT
\item For five diffenrent masses (100-500 GeV) 
\end{itemize}
\subsection{Preselection}
\begin{itemize}
\item Motivate different selection cuts
\item Reference DT search for most of them
\end{itemize}
\subsection{Main discriminating variables}
\begin{itemize}
\item dE/dx
\item pt
\item Show some MC signal bkg comparioson plots (only Wjets?)
\end{itemize}

%%%%%%%%%%%%%%%%%%%%%%%%%%%%%%%%%%%%%%%%%%%%%%%%%%%%%%%%%%%%%%%%%%%%%%%%%%%%%%%%%%%%%%%%%%%%%%%%%%%%%%%%%%%%%%%%%%%%%%%%%%%%%%%%%%%%%%%%%%%%%%%%%%%%%%%%%%%%%%%%%%%%%%%%%%%%%%%%%%%%
\section{Sources of backgrounds}
\label{sec:SourcesOfBackgrounds}
\begin{itemize}
\item Background consist of particles which make high energy deposits and are high pt
\item In general: Low background search
\end{itemize}
\subsection{Fake tracks}
\begin{itemize}
\item Definition of fake tracks
\item How can they fake the signal
\end{itemize}
\subsection{Muons}
\begin{itemize}
\item How can muons fake the signal
\end{itemize}
\subsection{Pions}
\begin{itemize}
\item How can pions fake the signal
\end{itemize}
\subsection{Electrons}
\begin{itemize}
\item How can electrons fake the signal
\end{itemize}
%%%%%%%%%%%%%%%%%%%%%%%%%%%%%%%%%%%%%%%%%%%%%%%%%%%%%%%%%%%%%%%%%%%%%%%%%%%%%%%%%%%%%%%%%%%%%%%%%%%%%%%%%%%%%%%%%%%%%%%%%%%%%%%%%%%%%%%%%%%%%%%%%%%%%%%%%%%%%%%%%%%%%%%%%%%%%%%%%%%%
%%%%%%%%%%%%%%%%%%%%%%%%%%%%%%%%%%%%%%%%%%%%%%%%%%%%%%%%%%%%%%%%%%%%%%%%%%%%%%%%%%%%%%%%%%%%%%%%%%%%%%%%%%%%%%%%%%%%%%%%%%%%%%%%%%%%%%%%%%%%%%%%%%%%%%%%%%%%%%%%%%%%%%%%%%%%%%%%%%%%
\section{Background estimation methods}
\label{sec:BackgroundEstimation}
\subsection{Fake background}
\subsection{Leptonic background}
\subsection{Systematic uncertainties}

%%%%%%%%%%%%%%%%%%%%%%%%%%%%%%%%%%%%%%%%%%%%%%%%%%%%%%%%%%%%%%%%%%%%%%%%%%%%%%%%%%%%%%%%%%%%%%%%%%%%%%%%%%%%%%%%%%%%%%%%%%%%%%%%%%%%%%%%%%%%%%%%%%%%%%%%%%%%%%%%%%%%%%%%%%%%%%%%%%%%
\section{Optimization of search sensitivity}
\label{sec:Optimization}
\begin{itemize}
\item Show plots
\item show table
\item Include NlostOuter here, too
\end{itemize}

%%%%%%%%%%%%%%%%%%%%%%%%%%%%%%%%%%%%%%%%%%%%%%%%%%%%%%%%%%%%%%%%%%%%%%%%%%%%%%%%%%%%%%%%%%%%%%%%%%%%%%%%%%%%%%%%%%%%%%%%%%%%%%%%%%%%%%%%%%%%%%%%%%%%%%%%%%%%%%%%%%%%%%%%%%%%%%%%%%%%
\section{Statistical Methods/ Limit setting}
\label{sec:LimitSetting}

%%%%%%%%%%%%%%%%%%%%%%%%%%%%%%%%%%%%%%%%%%%%%%%%%%%%%%%%%%%%%%%%%%%%%%%%%%%%%%%%%%%%%%%%%%%%%%%%%%%%%%%%%%%%%%%%%%%%%%%%%%%%%%%%%%%%%%%%%%%%%%%%%%%%%%%%%%%%%%%%%%%%%%%%%%%%%%%%%%%%
\section{Results}
\label{sec:Results}
\begin{itemize}
\item Data cutflowtable
\item Tables with results
\item One plot (4 bins: Prediction and data)
\end{itemize}

%%%%%%%%%%%%%%%%%%%%%%%%%%%%%%%%%%%%%%%%%%%%%%%%%%%%%%%%%%%%%%%%%%%%%%%%%%%%%%%%%%%%%%%%%%%%%%%%%%%%%%%%%%%%%%%%%%%%%%%%%%%%%%%%%%%%%%%%%%%%%%%%%%%%%%%%%%%%%%%%%%%%%%%%%%%%%%%%%%%%
\section{Interpretation}
\label{sec:Interpretation}
\subsection{Systematic uncertainties of simulated signal samples}
\subsection{Exclusion limits}
\begin{itemize}
\item 1-d limits
\item 2-d limits
\end{itemize}

