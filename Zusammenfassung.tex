\chapter*{Zusammenfassung}
Am CMS-Experiment am Large Hadron Collider (LHC) am CERN liegt der Hauptfokus auf der Suche nach Physik jenseits des Standardmodells und der Messung der Parameter des Standardmodells.
Daf\"{u}r ist es notwendig, wichtige Leistungsparameter bzw. Kenngr\"{o}{\ss}en des CMS-Detektors zu messen.
Die hier pr\"{a}sentierte Arbeit tr\"{a}gt in zweifacher Weise zu dem Physikprogramm des CMS-Experiments bei.

Im ersten Teil wird eine Suche nach neuer Physik, motiviert durch supersymmetrische Modelle mit fast masseentarteten Neutralinos und Charginos, vorgestellt.
Falls das leichteste Chargino nur wenig schwerer ist als das leichteste Neutralino, kann dies, wegen des verkleinerten Phasenraums, zu einer langen Lebenszeit des Charginos f\"{u}hren.
Die hier pr\"{a}sentierte Suche ist konzipiert f\"{u}r Modelle mit Charginolebensdauern von ungef\"{a}hr $\ctau\approx 1-30\cm$.
Bei diesen Lebensdauern können die Charginos schon sehr fr\"{u}h im Detektor zerfallen, sogar in den ersten Lagen des Spurdetektors.
Daher wird eine Erh\"{o}hung der Suchsensitivit\"{a}t im Vergleich zu fr\"{u}heren Suchen in zweifacher Weise versucht zu erreichen.
Zum einen durch die Einbindung von sehr kurzen rekonstruierten Spuren, zum anderen durch eine verbesserte Diskriminierung des Standardmodelluntergrunds durch die Variable \dedx, der Energieverlust pro Wegl\"{a}nge.
Die Suche nach hoch ionisierende, kurzen Spuren wird mit 8\tev Daten, die im Jahr 2012 am CMS-Experiment aufgenommen wurden, durchgef\"{u}hrt.
Der Untergrund wird hauts\"{a}chlich mit datengetriebener Methoden bestimmt und besteht zumeist aus falsch rekonstruierter Spuren, d.h. Spuren, die nicht durch ein einziges Teilchen verursacht wurden.
%Um die Sensitivit\"{a}t bez\"{u}glich unterschiedlicher Lebensdauern und Massen zu erh\"{o}hen wird die Suche in vier verschiedene Signalregionen durchgef\"{u}hrt.
Es konnte kein \"{U}berschuss beobachtet werden.
%Daher wird dieses Ergebnis verwendet, um den supersymmterischen Paramterraum einzuschr\"{a}nken.
Die Suche schlie{\ss}t mit diesem Ergebnis Modelle mit Charginomassen von 100\gev bis runter zu einer Lebensdauer von $\ctau=2\cm$ und Massen von 500\gev bis runter zu $\ctau=70\cm$ aus.
Damit konnten bestehende Ausschlussgrenzen best\"{a}tigt und st\"{a}rkere Ausschlussgrenzen zwischen 10-40\gev erzielt werden.

Im zweiten Teil dieser Arbeit wird die Messung der Jetimpulsaufloesung bei einer Schwerpunktsenergie von 8\tev vorgestellt.
Um die sehr gute Energieaufl\"{o}sung des elektromagnetischen Kalorimeter am CMS-Experiment auszunutzen, wird die Messung mithilfe von \GAMJET Ereignissen durchgef\"{u}hrt.
Dabei wird die Transversalimpulsbalance zwischen Jet und Photon, unter Vernachl\"{a}ssigung von weiterer Abstrahlung, ausgenutzt.
%In diesen Ereignissen haben das Photon und der Jet, unter Vernachl\"{a}ssigung von weiterer Abstrahlung, den gleichen Transversalimpuls und der gemessene Photonimpuls kann als Ma{\ss} f\"{u}r den Transversalimpuls des Jets ausgenutzt werden.
Durch die Weiterentwicklung der Methode, die den Einfluss der Richtung weiterer Jets in einem Ereignis auf die Jet-\pt-Response ber\"{u}cksichtigt, konnte eine gut funktionierde Methode sichergestellt werden.
Relative Aufl\"{o}sungsunterschiede zwischen gemessener und simulierter Daten liegen zwischen 7 und 20\%.
%Die Methode basiert auf fr\"{u}heren Messungen, ist jedoch weiterentwickelt worden, um konsistent mit dem Einfluss von der Richtung von weiterer Jets in einem Ereignis auf die Jet-\pt-Response umzugehen.
%Wegen dieser Weiterentwicklung, konnte eine gut funktionierde Methode sichergestellt werden.

\cleardoublepage
