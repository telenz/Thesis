\chapter*{Zusammenfassung}
Der Hauptfokus des CMS-Experiments am Large Hadron Collider (LHC) liegt auf der Suche nach Physik jenseits des Standardmodells und der Messung von Standardmodellparametern.
Daf\"{u}r ist es notwendig, wichtige Leistungsparameter bzw. Kenngr\"{o}{\ss}en des CMS-Detektors zu messen.
Die hier pr\"{a}sentierte Arbeit tr\"{a}gt in zweifacher Weise zu dem Physikprogramm des CMS-Experiments bei.

Im ersten Teil wird eine Suche nach neuer Physik, motiviert durch supersymmetrische Modelle mit fast masseentarteten Neutralinos und Charginos, vorgestellt.
Falls das leichteste Chargino nur wenig schwerer ist als das leichteste Neutralino, kann dies, wegen des verkleinerten Phasenraums, zu einer langen Lebenszeit des Charginos f\"{u}hren.
Die hier pr\"{a}sentierte Suche ist konzipiert f\"{u}r Modelle mit Charginolebensdauern von ungef\"{a}hr $\ctau\approx 1-30\cm$.
Bei diesen Lebensdauern können die Charginos schon sehr fr\"{u}h im Detektor zerfallen, sogar in den ersten Lagen des Spurdetektors.
Daher wird eine Erh\"{o}hung der Suchsensitivit\"{a}t im Vergleich zu fr\"{u}heren Suchen in zweifacher Weise versucht zu erreichen:
Zum einen durch die Einbindung von sehr kurzen rekonstruierten Spuren, zum anderen durch eine verbesserte Diskriminierung des Standardmodelluntergrunds durch den Energieverlust pro Wegl\"{a}nge.
Die Suche nach hoch ionisierende, kurzen Spuren wird mit 8\tev Daten, die im Jahr 2012 am CMS-Experiment aufgenommen wurden, durchgef\"{u}hrt.
Es konnte kein \"{U}berschuss \"{u}ber der Standardmodellvorhersage beobachtet werden und daher kann mit diesem Ergebnis der supersymmetrische Parameterraum eingeschr\"{a}nkt werden.
Die Suche schlie{\ss}t Modelle mit Charginomassen von 100\gev bis runter zu einer Lebensdauer von $\ctau=2\cm$ und SUSY Modelle mit Charginomassen von 500\gev bis runter zu $\ctau=70\cm$ aus.
Damit konnten bestehende Ausschlussgrenzen best\"{a}tigt und st\"{a}rkere Ausschlussgrenzen zwischen $10-40\gev$ erzielt werden.

Im zweiten Teil dieser Arbeit wird die Messung der Jetimpulsaufl\"{o}sung bei einer Schwerpunktsenergie von 8\tev vorgestellt.
Um die sehr gute Energieaufl\"{o}sung des elektromagnetischen Kalorimeter am CMS-Experiment zu nutzen, wird die Messung mithilfe von \GAMJET Ereignissen durchgef\"{u}hrt.
Dabei wird die Transversalimpulsbalance zwischen Jet und Photon, unter Vernachl\"{a}ssigung von weiterer Abstrahlung, ausgenutzt.
Durch die Weiterentwicklung der Methode, wird der Einfluss der Richtung weiterer Jets in einem Ereignis auf die Jet-\pt-Response konsistent ber\"{u}cksichtigt.
Relative Aufl\"{o}sungsunterschiede zwischen gemessener und simulierter Daten liegen zwischen 7 und 20\%.

\cleardoublepage
