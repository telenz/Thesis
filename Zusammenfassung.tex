\chapter*{Zusammenfassung}
Der Hauptfokus des CMS-Experiments am Large Hadron Collider (LHC) liegt auf der Suche nach Physik jenseits des Standardmodells und der Messung von Standardmodellparametern.
Daf\"{u}r ist es notwendig, wichtige Leistungsparameter bzw. Kenngr\"{o}{\ss}en des CMS-Detektors zu messen.
Die hier pr\"{a}sentierte Arbeit tr\"{a}gt in zweifacher Weise zu dem Physikprogramm des CMS-Experiments bei.

Im ersten Teil wird eine Suche nach neuer Physik, motiviert durch supersymmetrische Modelle mit fast masseentarteten Neutralinos und Charginos, vorgestellt.
Falls das leichteste Chargino nur wenig schwerer ist als das leichteste Neutralino, kann dies, wegen des verkleinerten Phasenraums, zu einer langen Lebenszeit des Charginos f\"{u}hren.
Die hier pr\"{a}sentierte Analyse ist f\"{u}r die Suche nach Modellen mit Charginolebensdauern von ungef\"{a}hr $\ctau\approx 1-30\cm$ konzipiert.
Bei diesen Lebensdauern können die Charginos schon sehr fr\"{u}h im Detektor zerfallen, sogar in den ersten Lagen des Spurdetektors.
Daher wird eine Verbesserung der Suchsensitivit\"{a}t im Vergleich zu bestehenden Suchen in zweifacher Weise zu erreichen versucht:
zum einen durch die Einbindung von sehr kurzen rekonstruierten Spuren, zum anderen durch eine verbesserte Unterdr\"{u}ckung des Standardmodelluntergrunds durch die Ber\"{u}cksichtigung des Energieverlusts pro Wegl\"{a}nge.
Diese Suche nach stark ionisierenden, kurzen Spuren wird mit 8\tev Daten, die im Jahr 2012 am CMS-Experiment aufgenommen wurden, durchgef\"{u}hrt.
Die Anzahl der beobachteten Ereignisse ist mit der Standardmodellvorhersage vereinbar, weshalb der supersymmetrische Parameterraum eingeschr\"{a}nkt werden kann.
Die Suche schlie{\ss}t Modelle mit Charginomassen von 100\gev bis zu einer unteren Lebensdauer von $\ctau=2\cm$ und SUSY Modelle mit Charginomassen von 500\gev bis zu einer unteren Lebensdauer von $\ctau=70\cm$ aus.
Damit k\"{o}nnen bestehende Ausschlussgrenzen best\"{a}tigt und um $10-40\gev$ bez\"{u}glich der Charginomasse verbessert werden.

Im zweiten Teil dieser Arbeit wird die Messung der Jetimpulsaufl\"{o}sung bei einer Schwerpunktsenergie von 8\tev vorgestellt.
Um die sehr gute Energieaufl\"{o}sung des elektromagnetischen Kalorimeters am CMS-Experiment zu nutzen, wird die Messung mithilfe von \GAMJET Ereignissen durchgef\"{u}hrt.
Dabei wird die Transversalimpulsbalance zwischen Jet und Photon, unter Vernachl\"{a}ssigung weiterer Abstrahlung, ausgenutzt.
Durch die in dieser Arbeit durchgef\"{u}hrte Weiterentwicklung der Methode wird der Einfluss der Richtung weiterer Jets in einem Ereignis auf die Jetimpulsaufl\"{o}sung konsistent ber\"{u}cksichtigt.
Relative Aufl\"{o}sungsunterschiede zwischen gemessener und simulierter Daten liegen zwischen 7 und 20\%.

\cleardoublepage
