%%%%%%%%%%%%%%%%%%%%%%%%%%%%%%%%%%%%%%%%%%%%%%%%%%%%%%%%%%%%%%%%%%%%%%%%%%%%%%%%%%%%%%%%%%%%%%%%%%%%%%%%%%%%%%%%%%%%%%%%%%%%%%%%%%%%%%%%%%%%%%%%%%%%%%%%%%%%%%%%%%%%%%%%%%%%%%%%%%%%%%%%%%%%%%%%%%%%%%%%%%%%%%%%%%%%%%%%%%%%%%%%%%%
\chapter{The Standard Model of particle physics}

%With the formulation of a relativistic quantum field theory and of spontaneous supersymmetry breaking by the Higgs mechanism, it was made possible to explain almost all observations of particle physics up to today.
The formulation of a relativistic quantum field theory and of spontaneous symmetry breaking by the Higgs mechanism, allowed to built a theory which is capable to explain almost all observations of particle physics until today~\cite{FIXME}.
This theory is known as the Standard Model of particle physics.
The existence of its last missing piece 
%(to be a full renormalisable and gauge invariant theory that can describe particle masses)
, the Higgs boson, could be proven at the LHC in the year 2012~\cite{FIXME}.

The Standard Model is a SU(3)xSU(2)xSU(1) non-abelian gauge theory, of which its symmetries are reduced ``after'' spontaneous symmetry breaking to SU(3)xU(1).
All particles that were found until today can be explained with it~\cite{FIXME}.
Furthermore, it is able to describe three of the four fundamental forces, the strong, weak and electromagnic force.

In the following, a small introduction to the theory and phenomenology of the Standard Model is given.
It is not meant as a complete description.
The reader is referred to \cite{FIXME}, for a thorough and extensive description.

\section{The Langrangian density}
The heart of every quantum field theory is the Lagrangian density with which it can be described.
For the Standard Model of particle physics, it is the following:
\begin{equation}
\mathcal{L} = 
\end{equation}
This is the smallest set of possible Lagrangian terms, that is renormilsable and contains all up to date known particles and the above mentioned gauge symmetries.





\section{The Higgs mechanism}
\section{Limitations of the Standard Model}

\chapter{Supersymmetry}
\section{Offering answers}

\section{The MSSM}
\subsection{The particle content of the MSSM}
\subsection{The phenomenological MSSM}

\section{Supersymmetry breaking}
\subsection{Gauge mediation}
\subsection{Gravity mediation}
\subsection{Anomaly-mediatied breaking}

\chapter{Long-lived particles}

\section{Long-lived charginos in the MSSM}
\subsection{Mechanisms}
\begin{itemize}
\item Mechanism of long lifetimes
\item Previous searches
\end{itemize}
