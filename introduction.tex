The last missing piece of the Standard Model of particle physics could be added by the discovery of the Higgs boson at the LHC in the year 2012~\cite{FIXME}.
By this, the formulation of the Standard Model is complete with all parameters (precisly) measured at collider experiments.
Up to now, the Standard Model could be validated in a variety of different measurements at particle colliders.

Nonetheless, there are doubts that the Standard Model is the ultimate theory of particle physics.
Experimental observations as well as theoretical considerations led to the persuasion that there exists physics beyond the Standard Model.
For instance, the observation of huge amounts of invisible matter in the universe cannot be exlplained from a particle perscpetive within the Standard Model.
Furthermore, a major theoretical concern is related to the occurence of quadratic divergencies in the calculation of the Higgs boson mass at higher radiative orders.
The Higgs boson is measured at a value of around 125\gev, which is considered as a very low value regarding the huge radiative corrections at the Planck scale ($10^{18}\gev$). 
This raises the questions, either what kind of mechanism is responsible for the stabilisation of the higgs boson mass at the electroweak scale or is the consideration of the Standard Model validity up to the Planck scale justified.
These shortcomings of the Standard Model as well as many other open questions led to strong efforts to formulate theories that go beyond the Standard Model of particle physics. 
One of these is able to solve the above mentioned problems by imposing a fully new symmetry into the Lagangrian formulation of particle physics, a so-called Supersymmetry.
This symmetry relates bosons and fermions by new fermionic generators and lead thus to the prediction of a supersymmetric partner particle for each of the particles contained in the Standard Model.
This would have drastic implications for the phenomenology of particle physics, since a doubling of the particle content is predicted.
The possibilty of searching for supersymmetric models are enourmous.

One can search either for very general signature, which are obvious to search for at hadron colliders strong production channel or to look into more exotic sectros well motivated by astrophycical observations.

In this PhD thesis, a search motivated by SUSY models that offer possible Dark Matter candidates is done.
In a certain parameter region, the lightest neutral and the lightest charged SUSY particle are almost degenerate leading to a long-lived heavy particle.
Thus a search for highly ionizing particles is done.
Furthermore, as esecially models with charged particles decayong early in the detector could not be excluded, the search concentrates on a search for short tracks in the tracker.

This signatures is very general and is sensitive to a variety of possible models but is motivated by a search for SUpersymmetry.

\begin{itemize}
\item The Standard Model - what it is and it is successful (Higgs discovery)
\item Why searching for beyond SM physics -  give one, two reasons
\item Explain why the signature I am looking for is very interesting
\item What are the main achievemnts ?
\item Find a good transition to the JER chapter
\item Why is this measurement of great importance
\item What are the main achievemnts
\item Give an overview how this thesis is structured
\end{itemize}
