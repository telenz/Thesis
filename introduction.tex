The last missing piece of the Standard Model of particle physics could be added with the discovery of the Higgs boson at the LHC in the year 2012~\cite{bib:Theory:CMS:HiggsObservation,bib:Theory:Atlas:HiggsObservation}.
By this time, the formulation of the Standard Model was complete and all of its parameters were (precisely) measured at particle physics experiments. 
Up to now, the Standard Model could be validated in a variety of different measurements at particle colliders.

Nonetheless, there are strong reasons to believe that the Standard Model is not the ultimate theory of particle physics.
Experimental observations as well as theoretical considerations led to the persuasion that there exists physics beyond the Standard Model.
For instance, the observation of Dark Matter cannot be explained from a particle perspective within the Standard Model.
Furthermore, a major theoretical concern is related to the occurrence of quadratic divergencies in the calculation of the Higgs boson mass at higher radiative orders.
The Higgs boson is measured at a value of around 125\gev, which is considered as a very low value regarding the huge radiative corrections at the Planck scale ($\sim 10^{19}\gev$). 
Therefore, the question arises, what kind of mechanism is responsible for the stabilisation of the Higgs boson mass at the electroweak scale. 
%This raises the questions, either what kind of mechanism is responsible for the stabilisation of the higgs boson mass at the electroweak scale or is the consideration of the Standard Model validity up to the Planck scale justified.
These shortcomings of the Standard Model as well as many other open questions led to strong efforts to formulate theories that go beyond the Standard Model of particle physics. 
One of these is able to solve the above mentioned problems by imposing a fully new symmetry into the Lagrangian formulation of particle physics, a so-called supersymmetry.
This symmetry relates bosons and fermions by new fermionic generators and lead thus to the prediction of a supersymmetric partner particle for each of the particles contained in the Standard Model.
This would have drastic implications for the phenomenology of particle physics, since a doubling of the particle content is predicted.


There is a variety of possibilities how to search for supersymmetric particles.
%One can search either for very general signature, which are obvious to search for at hadron colliders strong production channel or to look into more exotic sectros well motivated by astrophycical observations.
In this PhD thesis, a search motivated by supersymmetric models with nearly mass-degenerate lightest (\chiO) and next-to lightest (\chipm) supersymmetric particles is presented.
A small mass splitting between the two particles can lead to a long-lifetime of \chipm.
Since, the next-to lightest supersymmetric particle is charged in these SUSY models, it can appear as reconstructed track in the inner tracking system when reaching the CMS detector. 
Since, the masses of the supersymmetric particles are in general higher than their Standard Model partners, \chipm can be heavy and can therefore deposit much higher energies in the tracker.
By this, a good discrimination against the Standard Model background of minimally ionising particle can be achieved.
Furthermore, the here presented analysis concentrates on supersymmetric models not yet excluded by other searches. 
Therefore, it aims at targeting models at \chipm lifetimes leading to rather short tracks in the tracker. 
The analysis strategy consists therefore in searching for highly ionising, short tracks.
It makes use of 19.7\fbinv of data, taken in the year 2012 at a centre-of-mass energy of 8\tev.\\
%This signatures is very general and is sensitive to a variety of possible models but is motivated by a search for SUpersymmetry.

In the second part of the thesis, a measurement of the jet transverse-momentum resolution is performed.
The knowledge of the jet \pt resolution is a crucial ingredient for many analyses at CMS, \eg~\cite{FIXME}.
In order to exploit the good calorimeter energy resolution of the CMS experiment, the measurement is performed using \GAMJET events, where the photon energy can be used as a measure for the true jet transverse momentum.
The applied method is grounded on earlier measurements but is further developed in order to account for the influence of the direction of further jets in the event on the jet transverse-momentum response.
The most important application of the measurement of the jet transverse-momentum response is the adjustment of the resolution in simulation to the one measured in data.
Therefore , the results are presented as data-to-simulation resolution scale factors.\\

The thesis is structured into six main parts.
First, in Part~2 the theoretical foundations are introduced.
It comprises an introduction into the Standard Model of particle physics as well as of its supersymmetric extension and introduces into the phenomenology of long-lived particles in supersymmetric models.
In Part~3, the experimental setup is  presented on which the search for highly ionising short tracks and the measurement of the jet transverse-momentum resolution is based.
This contains an introduction of the Large Hadron Collider and the CMS experiment.
Furthermore, the event reconstruction and particle identification is discussed.
Finally, a very short introduction into the techniques of event simulation is given.
In Part~4, the search for highly ionising, short tracks is presented performed within this thesis.
Part~5 presents the measurement of the jet transverse-momentum resolution with \GAMJET events with $pp$-collision data from the CMS experiment.
Finally, Part~6 concludes and summarises the presented methods and results.

