\noindent With the discovery of the Higgs boson at the LHC in the year 2012, the last missing piece of the Standard Model of particle physics was found~\cite{bib:Theory:CMS:HiggsObservation,bib:Theory:Atlas:HiggsObservation}.
Thus, all particles contained in the Standard Model are discovered and all of its parameters are measured, many of them with accuracies at the per-mille level.
Up to now,  the Standard Model has been tested at many particle physics experiments and has proven its ability to explain -~and even predict~- experimental results in a remarkable way.


Nonetheless, there are strong reasons to believe that the Standard Model is not the ultimate theory of particle physics.
Experimental observations as well as theoretical considerations have led to the belief that there exists physics beyond the Standard Model.
For instance, the observation of Dark Matter cannot be explained within the Standard Model since no suitable Dark Matter candidate is contained.
From a theoretical point of view, a major concern is related to the occurrence of quadratic divergencies in the calculation of the Higgs boson mass at higher radiative orders.
The Higgs boson mass is measured at a value of around 125\gev\footnote{Throughout this thesis, natural units ($\hbar = c = 1$) are used.}, which is considered very low regarding the huge radiative corrections at the Planck scale ($\sim 10^{19}\gev$). 
This raises the question of what kind of mechanism is responsible for the stabilisation of the Higgs boson mass at the electroweak scale. 
Among others, these shortcomings of the Standard Model have led to strong efforts to develop theories that go beyond the Standard Model of particle physics. 

One of these theories is able to solve the above mentioned problems by imposing a new symmetry into the Lagrangian formulation of particle physics, a so-called supersymmetry (SUSY).
This symmetry relates bosons and fermions by new fermionic generators and leads to the prediction of a supersymmetric partner particle for each of the particles contained in the Standard Model.
This could have drastic implications for the phenomenology of particle physics, since a doubling of the particle content is predicted.
Therefore, a variety of searches for supersymmetric particles has been performed at many particle physics experiments.\\

This PhD thesis presents a search for supersymmetric particles in 19.7 \fbinv of data, taken in the year 2012 at a centre-of-mass energy of 8\tev at the CMS detector. 
The search is motivated by supersymmetric models with nearly mass-degenerate lightest (\chiO) and next-to lightest (\chipm) supersymmetric particles that have not yet been targeted by existing SUSY searches.
A small mass splitting between the two particles can lead to a long lifetime of the next-to lightest supersymmetric particle \chipm because of phase space suppression.
The charged \chipm can therefore appear as a reconstructed track in the inner tracking system of the CMS detector.
At rather low \chipm lifetimes, the \chipm potentially decays inside the tracker and the reconstructed track can be very short.  
Furthermore, since the masses of supersymmetric particles are in general higher than their Standard Model partners, \chipm can be heavy and can therefore deposit much higher energies in the tracker compared to minimally ionising Standard Model particles.
Therefore, the analysis strategy of the here presented analysis is to search for highly ionising, short tracks.
It is the first analysis at CMS that incorporates tracks with down to three measurement and that makes use of the energy information of the silicon pixel tracker, which has been subject to an energy calibration within this thesis.\\

The second research objective of this thesis is a measurement of the jet transverse-momentum resolution at a centre-of-mass energy of 8\tev at CMS.
The knowledge of the jet \pt resolution is a crucial ingredient for many analyses at CMS, \eg the measurement of the dijet cross section~\cite{bib:CMS:QCD_measurements} and searches for physics beyond the Standard Model that rely on a good understanding of missing energy originating from wrongly measured jets~\cite{bib:CMS:RA2_8TeV}.

In order to exploit the good energy resolution of the electromagnetic calorimeter at the CMS experiment, the measurement is performed using \GAMJET events.
Due to the transverse momentum balance of \GAMJET events in the absence of further jet activity, the photon energy can be used as a measure for the true jet transverse momentum.
The applied method is based on earlier measurements~\cite{bib:CMS:JERCPaper_2011,CMS:PAS:JETResolution_7TeV} but is further developed within this thesis in order to consistently account for the influence of additional jet activity on the jet transverse-momentum response.\\

\noindent The thesis is structured into six main parts.
\begin{description} 
%\setlength\itemsep{1em}
\item \textbf{\hyperref[part:Theory]{Part~2:}} This part summarises the theoretical foundations, comprising an introduction to the Standard Model of particle physics as well as to its supersymmetric extensions. A special focus is on the theoretical description and phenomenology of long-lived particles in supersymmetric models. 
\item \textbf{\hyperref[part:Experiment]{Part~3:}} Within this part, the experimental setup is  presented, including an introduction to the Large Hadron Collider and the CMS experiment as well as a description of the algorithms used for event reconstruction and particle identification at CMS. Finally, a short introduction into the techniques of event simulation is given.
\item \textbf{\hyperref[part:analysis]{Part~4:}}  In this part, the search for highly ionising, short tracks is presented. It starts with a motivation and an outline of the general search strategy. Afterwards, the calibration of the silicon pixel tracker is described and its impact on the search is discussed. Subsequently, the event selection is described and the background estimation methods are introduced. Finally, the results are presented and interpreted in the context of supersymmetric models with long-lived \chipm. The last chapter of this part is devoted to a conclusion and discussion of the most important findings.
\item \textbf{\hyperref[part:resolution]{Part~5:}} This part presents the measurement of the jet transverse-momentum resolution in \GAMJET events recorded at CMS at $\sqrt{s}=8\tev$. It starts with a motivation and a presentation of the general approach of the measurement. The introduction of the event selection is followed by a thorough description of the methodology. Afterwards, the systematic uncertainties are discussed. Finally, the results are presented, followed by a conclusion and discussion.
\item \textbf{\hyperref[part:Summary]{Part~6:}} This part concludes and summarises the most important results of this thesis. 
\end{description}

