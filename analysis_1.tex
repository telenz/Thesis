%%%%%%%%%%%%%%%%%%%%%%%%%%%%%%%%%%%%%%%%%%%%%%%%%%%%%%%%%%%%%%%%%%%%%%%%%%%%%%%%%%%%%%%%%%%%%%%%%%%%%%%%%%%%%%%%%%%%%%%%%%%%%%%%%%%%%%%%%%%%%%%%%%%%%%%%%%%%%%%%%%%%%%%%%%%%%%%%%%%%
\FloatBarrier
\chapter{Motivation}
\label{sec:Motivation}
Supersymmetry is able to offer solutions to many unexplained phenomena in astrophysics and can solve many of the shortcomings of the Standard Model of particle physics (see Section~\ref{FIXME}).
While SUSY has been studied at previous particle colliders including Tevatron and LEP~\cite{bib:Tevatron:SUSY_results,bib:LEP:SUSY_results}, the LHC with its high centre-of-mass energy offers a unique opportunity to investigate SUSY models with high sparticle masses that were not accessible in previous experiments.

Therefore, a variety of searches were hunting for SUSY during the run\,I of the LHC in 2011 and 2012.
Proton-proton collision data from the CMS and ATLAS experiments were analysed with a strong focus on the search for SUSY in the strong production sector (\eg~\cite{bib:CMS:RA2_8TeV,bib:CMS:MT2_8TeV,bib:ATLAS:JetPlusMET_8TeV}).
As a consequence, wide, previously unexplored regions of SUSY parameter space are already excluded.
However, due to the unknown mechanism of supersymmetry breaking, the most general parametrisation of the Minimal Supersymmetric Standard Model (MSSM) introduces over 100 new parameters and thus opens up an incredibly large phenomenological space. Therefore, SUSY models can lead to a plethora of possible signatures at particle colliders. \\

(I think it is wrong - What motivation??)
%Among more ``exotic'' SUSY scenarios are models with compressed spectra, where two or more particles are nearly mass-degenerate.
%Especially scenarios with a nearly mass-degenerate lightest chargino (\chipm) and lightest neutralino (\chiO) are very interesting from a theoretical and cosmological perspective as they can help to explain the sources of the relic den%sity~\cite{bib:Moroi:DarkMatter_1999,bib:Hisano:DarkMatter_2005,bib:Ibe:DarkMatter_2015}.
%While it is not possible to explain the full relic density with thermally produced neutralinos for m$_{\chiO}\lesssim 2.9\tev$~\cite{bib:Moroi:DarkMatter_2013}, neutralinos can still be the dominant part if they are non-thermally produ%ced via the decay of a long-lived particle such as a wino-like chargino.
%The enhanced annihilation cross section (called Sommerfeld enhancement) into $WW$- , $ZZ$- or $ff$-pairs for a wino-like dark matter candidate leads to an underprediction of the relic density if the neutralino and chargino masses are t%oo small~\cite{bib:Hisano:DarkMatter_2003}.
%This underprediction can be cured, however, if there is an additional non-thermal production of dark matter that is caused by the decay of a long-lived chargino.
%In Supersymmetry, such a mass-degeneracy naturally occurs in case of wino-like neutralinos and charginos, since the mass gap between $W_{3}$ and $W_{1/2}$ is fully determined by higher loop corrections (see Section~\ref{FIXME}).

%Among more ``exotic'' SUSY scenarios are models with compressed spectra, where two or more particles are nearly mass-degenerate.
%Especially scenarios with a wino-like 
%Especially scenarios with a nearly mass-degenerate lightest chargino (\chipm) and lightest neutralino (\chiO) are very interesting from a theoretical and cosmological perspective as they can help to explain the sources of the relic den%sity~\cite{bib:Moroi:DarkMatter_1999,bib:Hisano:DarkMatter_2005,bib:Ibe:DarkMatter_2015}.
%While it is not possible to explain the full relic density with thermally produced neutralinos for m$_{\chiO}\lesssim 2.9\tev$~\cite{bib:Moroi:DarkMatter_2013}, neutralinos can still be the dominant part if they are non-thermally produ%ced via the decay of a long-lived particle such as a wino-like chargino.
%The enhanced annihilation cross section (called Sommerfeld enhancement) into $WW$- , $ZZ$- or $ff$-pairs for a wino-like dark matter candidate leads to an underprediction of the relic density if the neutralino and chargino masses are t%oo small~\cite{bib:Hisano:DarkMatter_2003}.
%This underprediction can be cured, however, if there is an additional non-thermal production of dark matter that is caused by the decay of a long-lived chargino.
%In Supersymmetry, such a mass-degeneracy naturally occurs in case of wino-like neutralinos and charginos, since the mass gap between $W_{3}$ and $W_{1/2}$ is fully determined by higher loop corrections (see Section~\ref{FIXME}).

Among more ``exotic'' SUSY scenarios are models with compressed spectra, where two or more particles are nearly mass-degenerate.
Especially scenarios with a nearly mass-degenerate lightest chargino (\chipm) and lightest neutralino (\chiO) are very interesting from a theoretical and cosmological perspective, as will be explained immediately.
In R-Parity conserving Supersymmetry, such a mass-degeneracy naturally occurs in case of wino-like neutralinos and charginos, since the mass gap is fully determined by higher loop corrections (see Section~\ref{FIXME}) and thus very small.
A wino-like neutralino on the other hand is attractive from an cosmological perspective, as it can help to explain the sources of the relic density~\cite{bib:Moroi:DarkMatter_1999,bib:Hisano:DarkMatter_2005,bib:Ibe:DarkMatter_2015}.
While it is not possible to explain the full relic density with thermally produced neutralinos for m$_{\chiO}\lesssim 2.9\tev$~\cite{bib:Moroi:DarkMatter_2013}, neutralinos can still be the dominant part if they are non-thermally produced via the decay of a heavy and long-lived particle (moduli field).
The enhanced annihilation cross section (called Sommerfeld enhancement) into $WW$- , $ZZ$- or $ff$-pairs for a wino-like dark matter candidate leads to an underprediction of the relic density if the neutralino mass is too small~\cite{bib:Hisano:DarkMatter_2003}.
This underprediction can be cured, however, if there is an additional non-thermal production of dark matter as explained before.
If the wino mass paramater is much smaller than the bino and higgsino mass parameters, both, the lightest neutralino and the lightest chargino are wino-like and almost mass-degenerate.
Apart from heaving a viable Dark Matter candidate, it would also lead to the prediction of a charged sparticle, that could be detected in the CMS experiment.



SUSY scenarios with nearly mass-degenerate particles have two distinctive phenomenological properties that require a very different search strategy compared to general SUSY searches. 
First, if the chargino and the neutralino are almost mass-degenerate ($\Delta m \lesssim 200\mev$), the remaining decay product (\eg a pion) is very soft in \pt, making it hard to detect.
Second, the chargino is long-lived due to phase space suppression (see Section~\ref{sec:FIXME.Theory:Lifetimes}) and may traverse several detector layers before decaying. \\
%This allows to exploit track charateristics of the chargino like the high ionisation losses due to its high mass.

Although supersymmetric models with nearly mass-degenerate \chipm and \chiO lead to exotic signatures with long-lived charginos and soft decay products, existing SUSY searches at CMS can in principle be sensitive to these models. 
The exclusion power of existing SUSY searches can be assessed by interpreting their results in terms of the fraction of excluded parameter points in the phenomenological MSSM (see Section~\ref{theorySUSY} for an introduction to the pMSSM). 
The results of such a study which has been performed in \cite{bib:CMS:DT_8TeV} are shown in Figure~\ref{fig:pMSSMplot}. 
\begin{figure}[!t]
  \centering 
  \begin{tabular}{c}
    \includegraphics[width=0.75\textwidth]{figures/analysis/pMSSM_vs_ctau.pdf}
  \end{tabular}
  \caption{The number of excluded pMSSM points at 95\% C.L. (upper part) and the fraction of excluded pMSSM points (bottom part) vs. the chargino lifetime for different CMS searches.
           Red area: the search for long-lived charged particles \cite{bib:CMS:HSCP_8TeV},
           Purple area: the search for disappearing tracks  \cite{bib:CMS:DT_8TeV},
           Blue area: a collection of various general SUSY searches \cite{bib:CMS:pMSSMinterpretation_7TeV_PAS}
           The black line indicates the unexcluded pMSSM parameter points.
           The sampling of the parameter space points was done according to a prior probability density function which takes pre-LHC data and results from indirect SUSY searches into account (see \cite{bib:CMS:HSCPReinterpreation_PAS} for further details).
           Taken from: \cite{bib:pMSSMplot_source_from_DT}.}
  \label{fig:pMSSMplot}
\end{figure}
It can be seen that general SUSY searches (blue area) are sensitive to shorter chargino lifetimes ($c\tau \lesssim 10\cm$). 
Due to technical reasons\footnote{The pMSSM interpretation relied on the use of fast simulation techniques which are not capable of simulating charginos with lifetimes $\ctau>1\cm$.}, the general SUSY searches were never interpreted in the context of SUSY models with longer chargino lifetimes. 
Two existing searches, the search for long-lived charged particles~\cite{bib:CMS:HSCP_8TeV} and the search for disappearing tracks~\cite{bib:CMS:DT_8TeV} focus on long and intermediate chargino lifetimes, respectively. 
These two searches (purple and red areas) are sensitive to chargino lifetimes of $\ctau \gtrsim 35\cm$.
Taken together, the existing searches exclude a large fraction of pMSSM points at different chargino lifetimes. 
However, there is a gap between the general SUSY searches and the search for disappearing tracks that is not accessible by any of the existing searches.\\

The here presented analysis aims at targeting this gap by optimising the search strategy for charginos with intermediate lifetimes of $10\cm \lesssim c\tau \lesssim 40\cm$. 
The targeted optimisation strategy is a combination of the strategies used in the search for long-lived charged particles~\cite{bib:CMS:HSCP_8TeV} and the search for disappearing tracks~\cite{bib:CMS:DT_8TeV}.
While in~\cite{bib:CMS:HSCP_8TeV}, the high ionisation losses of hypothetical new massive particles is exploited, it does not take into account whether its reconstructed track is disappearing.
In~\cite{bib:CMS:DT_8TeV}, the disappearance of the track is utilised but it does not incorporate the large ionisation losses into the search.
Additionally, neither of the search does take into account the possible very short tracks of a early decaying chargino.
Thus, the here presented search is the first analysis at CMS combining the two signature properties that are highly distinctive for charginos with intermediate lifetimes: 
first, the characteristically high ionisation losses of heavy charginos;
second, short reconstructed tracks due to chargino decays early in the detector. 

The associated challenges and the general search strategy of this analysis will be presented in the next section.

%%%%%%%%%%%%%%%%%%%%%%%%%%%%%%%%%%%%%%%%%%%%%%%%%%%%%%%%%%%%%%%%%%%%%%%%%%%%%%%%%%%%%%%%%%%%%%%%%%%%%%%%%%%%%%%%%%%%%%%%%%%%%%%%%%%%%%%%%%%%%%%%%%%%%%%%%%%%%%%%%%%%%%%%%%%%%%%%%%%%
%%%%%%%%%%%%%%%%%%%%%%%%%%%%%%%%%%%%%%%%%%%%%%%%%%%%%%%%%%%%%%%%%%%%%%%%%%%%%%%%%%%%%%%%%%%%%%%%%%%%%%%%%%%%%%%%%%%%%%%%%%%%%%%%%%%%%%%%%%%%%%%%%%%%%%%%%%%%%%%%%%%%%%%%%%%%%%%%%%%%
\FloatBarrier
\chapter{General search strategy}
\label{sec:GeneralSearchStrategy}

At the LHC, there are several possible chargino production channels. 
Chargino pairs can be produced through a photon or a $Z$-boson exchange. 
The chargino then decays via a virtual $W$-boson to the lightest neutralino and a fermion pair (\eg a pion). 
This process is illustrated in the Feynman diagram in Fig.~\ref{fig:FeynmanDiagram}.
Other possible chargino pair production channels include the exchange of a supersymmetric Higgs boson or a t-channel squark exchange (Fig.~\ref{fig:FeynmanDiagramProductionCharginoPair}).

Apart from pair production, charginos can be produced via the chargino neutralino production channel. 
On tree-level, there exist two production mechanisms: the s-channel $W$-boson exchange and the t-channel squark exchange (Fig.~\ref{fig:FeynmanDiagramProductionCharginoNeutralino}).

\begin{figure}[!b]
  \centering 
  \begin{tabular}{c}
    \includegraphics[width=0.75\textwidth]{figures/analysis/ChiChi_ProductionAndDecay.pdf}
  \end{tabular}
  \caption{Feynman diagram of chargino pair production via gamma or $Z$-boson exchange and the subsequent decay via a virtual $W$-boson.}
  \label{fig:FeynmanDiagram}
\end{figure}

\begin{figure}[!h]
  \centering 
  \begin{tabular}{c}
    \includegraphics[width=0.33\textwidth]{figures/analysis/ChiChi_GammaZ.pdf}
    \includegraphics[width=0.33\textwidth]{figures/analysis/ChiChi_Scalar.pdf}
    \includegraphics[width=0.33\textwidth]{figures/analysis/ChiChi_Squark.pdf}
  \end{tabular}
  \caption{Main tree-level diagrams for chargino pair production.}
  \label{fig:FeynmanDiagramProductionCharginoPair}
\end{figure}

\begin{figure}[!h]
  \centering 
  \begin{tabular}{c}
    \includegraphics[width=0.33\textwidth]{figures/analysis/ChiChi0_WBoson.pdf}
    \includegraphics[width=0.33\textwidth]{figures/analysis/ChiChi0_Squark.pdf}
  \end{tabular}
  \caption{Main tree-level diagrams for chargino neutralino production.}
  \label{fig:FeynmanDiagramProductionCharginoNeutralino}
\end{figure}
Alternatively, charginos can be produced via strong production modes, \ie in cascade decays of new heavy particles, such as gluinos or squarks.
In the here presented search, the focus is, however, put on the electroweak production channels: chargino-pair and chargino-neutralino production.\\


When searching for supersymmetric models with long-lived \chipm, the strategy is of course highly dependent on the actual lifetime of the chargino. 
For long lifetimes, the chargino can reach the muon chambers and can be reconstructed as a muon even despite a longer time-of-flight \cite{bib:CMS:HSCP_7TeV}. 
For lower lifetimes, the chargino can already decay inside the detector (\eg the tracker), and can hence not be reconstructed as a muon but leads to an isolated, potentially disappearing track in the tracker. 
The detector signatures of these two scenarios are visualised in Fig.~\ref{fig:CharginoPaiEventDisplay}, where simulated chargino-chargino events are shown in a cross-sectional view of the CMS detector.
\begin{figure}[!t]
  \centering 
  \begin{tabular}{c}
  \begin{subfigure}{0.31\textwidth}
    \frame{\includegraphics[width=0.99\textwidth]{figures/analysis/MotivationAndGeneralSearchStrategy/CharginoPairEvent_ctau_10000cm_lumi_1_event_12015.png}}
      \caption{}
  \end{subfigure} 
  \begin{subfigure}{0.31\textwidth}
    \frame{\includegraphics[width=0.99\textwidth]{figures/analysis/MotivationAndGeneralSearchStrategy/CharginoPairEvent_ctau_50cm_lumi_1_event_11024.png}}
      \caption{}
  \end{subfigure} 
  \begin{subfigure}{0.31\textwidth}
      \frame{\includegraphics[width=0.99\textwidth]{figures/analysis/MotivationAndGeneralSearchStrategy/CharginoPairEvent_ctau_50cm_lumi_1_event_11024_Zoom.png}}
      \caption{}
  \end{subfigure} 
  \end{tabular}
  \caption{Visualisation of possible signatures of a chargino pair produced with a lifetime of $\ctau = 10\,\text{m}$ (a) and a lifetime of $\ctau = 0.5\,\text{m}$ (b and c). 
           The muon chambers are the outer layers of the detector and are depicted as red boxes.
           The black lines represent the reconstructed chargino tracks.
           The right picture is a zoom of the picture in the middle. Here, only the cross-section of the tracker (green wavy lines for the strip and grey lines for the pixel) is displayed. The red arrow shows the missing transverse energy in the event.
           The red (blue) towers correspond to the energy deposition in the ECAL (HCAL).} 
  \label{fig:CharginoPaiEventDisplay}
\end{figure}
In the left picture of Fig.~\ref{fig:CharginoPaiEventDisplay}, both charginos are reconstructed as muons, which can be seen in the energy deposition in the muon chambers.
In the middle and right pictures both charginos have a lower lifetime of $\ctau=0.5\,$m and thus are only visible as tracks in the tracker, where both trajectories end inside the silicon strip tracker.
Since this analysis targets a search for Supersymmetry with charginos of lifetimes between $\ctau \approx 10\,\text{cm} - 40\,\text{cm}$, the charginos decay rather early in the detector, even in the inner layers of the tracker.
%Thus, the signature of chargino events consists of isolated, short tracks that have large ionisation losses due to high chargino masses and the signatures of the decay products, \ie of a neutralino and a fermion pair. 
Thus, the signature of chargino events consists of isolated, short tracks and the signatures of the decay products, \ie of a neutralino and a fermion pair. 

In case of R-parity conservation one of the chargino decay products, the neutralino, is stable and weakly interacting, thus traversing the detector without leaving any further signature.
%The missing transverse energy of the neutralino is balanced by the missing transverse energy of the second produced SUSY particle.
%This is either a neutralino or the decay products of the chargino in events with chargino pairs. 

The signature of the other decay product, the fermion pair, can in principle be used to select chargino events. 
However, for mass-degenerate charginos, it can be very hard or even impossible to detect these fermions as will be explained in detail in the next paragraph.

First of all, the fermionic decay product (\eg a pion) can usually not be reconstructed because it does not origin from the primary vertex.
Secondly, it is very low in momentum because of the mass-degeneracy between \chipm and \chiO.
The typical momentum of a pion originating from a chargino to neutralino decay in the \chipm rest frame is of the order 
\begin{equation}
p_{\pi}\sim \sqrt{m_{\chi^{\pm}_1}-m_{\chi^{0}_1}-m_{\pi}}.
\end{equation}
%As the \chipm is rather slow for higher masses this is a good approximation also in the detector rest frame.
For a mass gap between \chipm and \chiO of $\Delta m=150\, \mev$, the pt distribution of the resulting pion peaks \mbox{at $\sim$ 100\,\mev} and ends at \mbox{\pt $\sim 400\,$\mev} (Fig.~\ref{fig:ptOfPions}).
\begin{figure}[!t]
  \centering 
  \begin{tabular}{c}
    \includegraphics[width=0.49\textwidth]{figures/analysis/PtOfPions.pdf}
  \end{tabular}
  \caption{Transverse momentum distribution of pions coming from chargino decay into a neutralino with a mass gap of 150\mev.}
  \label{fig:ptOfPions}
%\vspace{90pt}
\end{figure} 

If the transverse momentum of a particle is very low, the particle trajectory is much more bended compared to a particle with higher \pt (see Fig.~\ref{fig:KinkedTrack} for illustration).
Due to this bending, the track reconstruction efficiency of particles with a transverse momentum below 1\gev decreases rapidly, reaching around 40\% for isolated pions with a \pt of 100\mev~\cite{bib:CMS:tracking_8TeV}. 
Furthermore, for pions that are not produced in the primary vertex, this reconstruction efficiency will be even smaller.
It is therefore impossible to rely on a reconstruction of the fermionic chargino decay products in this analysis.
\begin{figure}[!b]
  \centering 
  \begin{tabular}{c}
    \frame{\includegraphics[width=0.4\textwidth]{figures/analysis/MotivationAndGeneralSearchStrategy/BendedPionTrack.png}}
    %\frame{\includegraphics[width=0.30\textwidth]{figures/analysis/KinkedTrackZoom_Quadrat.png}}
  \end{tabular}
  \caption{Cross-sectional view of the tracker (silicon strip (silicon pixel) tracker layers are illustrated with green (grey) lines) and a simulated chargino track (black line) decaying to a pion (bended blue line) with a \pt of $\sim$85\mev and a neutralino (not visible).}
  \label{fig:KinkedTrack}
\end{figure} 

In summary, since an early decaying chargino is not reconstructed as a PF particle, the event signature of a chargino-pair or a chargino-neutralino event consists only of one (or two) - potentially - disappearing track. 
Such a signature is very difficult to detect, especially since CMS doesn't offer a dedicated track trigger so that triggering on the chargino track is impossible.


In order to search for such signatures, one therefore needs to trigger on other, less obvious properties of chargino events. 
This analysis takes advantage of higher order contributions to the Feynman diagrams shown in Figs.~\ref{fig:FeynmanDiagramProductionCharginoPair} and~\ref{fig:FeynmanDiagramProductionCharginoNeutralino}, resulting in initial state radiation (ISR).
If the initial quarks radiate a high \pt gluon, the resulting jet can be detected and can offer a possibility to search for events with nothing more than isolated tracks.
Furthermore, the non-detection of the chargino's decay products plus a high \pt ISR jet lead to missing transverse energy (MET) in the event. 
Exploiting these two circumstances, it is possible to detect chargino-pair or chargino-neutralino events with the help of Jet+MET triggers.

Since Jet+MET triggers are not very specific for chargino events, it is important to identify further track properties that can be used to select chargino candidates.
One distinctive property of charginos compared to SM particles is their high mass. 
Therefore, charginos can be identified by selecting high \pt tracks. 
Furthermore, the energy loss per path length (\dedx) depends quadratically on the particle's mass for low velocities ($0.2<\beta\gamma<0.9$):
\begin{equation*}
\langle\frac{dE}{dx}\rangle = K \frac{m^2}{p^2} +C
\end{equation*}
Therefore, \dedx constitutes a very nice discriminating variable for massive particles like charginos against SM particles.
The selection of chargino events in this analysis thus relies on the selection of isolated high \pt tracks with high \dedx values. 

If the chargino decays before it has crossed the full pixel and strip detector, the associated track is disappearing. 
For low lifetimes, the tracks can be very short and can have only a few hits in the detector. 
In order to reconstruct a particle's trajectory, a minimum of three hits are required since defining a helical path requires five parameters (see \cite{bib:CMS:tracking_8TeV}). 
A specific challenge for this analysis is hence the combination of searching for short tracks and utilising the measurement of the energy deposition of the chargino. 
For very short tracks, eventually only passing the first couple of layers of the whole tracker system, the pixel tracker information becomes very important. 
Therefore, an accurate energy measurement in the pixel system is of great importance to this analysis. 
However, no other CMS analysis has used the energy information of the pixel tracker so far.
This analysis thus requires a thorough study of the quality of the pixel energy calibration and, potentially, a recalibration in case the pixel energy calibration is not sufficient.



\section{Comparison to earlier searches}
As already mentioned before, there are two analyses at CMS at $\sqrt{s}=8\,\tev$ with 20\,fb$^{-1}$ data that search for intermediate lifetime charginos, the search for long-lived charged \mbox{particles \cite{bib:CMS:HSCP_8TeV}} and the search for disappearing tracks \cite{bib:CMS:DT_8TeV}.
The here presented analysis aims at achieving an increase in sensitivity towards shorter lifetimes compared to the earlier analyses in a twofold way.
First, the selection is optimised for the inclusion of very short tracks.
Second, the inclusion of the variable \dedx is used to increase the search sensitivity compared to \cite{bib:CMS:DT_8TeV}.\\

In \cite{bib:CMS:HSCP_8TeV}, a minimum number of eight hits were required for every track, whereas \cite{bib:CMS:DT_8TeV} required a minimum of seven hits.
This can be very inefficient for shorter lifetimes, where most of the charginos already decay shortly after the pixel tracker.
In Fig.~\ref{fig:NHits_2Signal_noSelection_normalized} (left), the normalised distribution of the number of measurements (\nhits) of chargino tracks is shown. 
It can be seen, that \nhits peaks at the minimal possible value needed for track reconstruction of $\nhits=3$ for lower lifetimes.
%For higher lifetimes ($\ctau=50\cm$) the distribution shifts to higher values with a second peak at $\nhits\sim17$.
For a lifetime of $\ctau=50\cm$, a second peak at $\sim$17 hits appears corresponding to the number of measurements when crossing all pixel barrel (3) and strip inner and outer barrel (6 from stereo and 8 from normal) layers.
However, a notable fraction of $\sim$ 40\% of chargino tracks still has a number of measurements of $\nhits<8$. 

It should be also mentioned, that the track reconstruction efficiency is sufficient for short chargino tracks, such that a loosening of the \nhits requirement is expected to be really improving the signal acceptance.
The track reconstruction efficiency for different chargino decay points is depicted in Fig.~\ref{fig:NHits_2Signal_noSelection_normalized} (right).
For very short tracks ($\nhits=3$) the efficiency is still around 20\%.\\
\begin{figure}[!b]
  \centering 
  \begin{tabular}{c}
  \includegraphics[width=0.49\textwidth]{figures/analysis/MotivationAndGeneralSearchStrategy/htrackNValid_log_chiTracksnoSelection.pdf}
  %\includegraphics[width=0.49\textwidth]{figures/analysis/htrackNValid_log_chiTracksnoSelection_m500GeV.pdf}
  \includegraphics[width=0.49\textwidth]{figures/analysis/MotivationAndGeneralSearchStrategy/RecoEffTracksZoom.png}
  \end{tabular}
  \caption{Left: Number of measurements in the tracker system \nhits for four different signal lifetimes.
           Right: Probability to reconstruct a track (z) in dependency of the chargino's decay point (x and y).
           More information on the generation of the simulated signal samples can be found in Section~\ref{sec:SignalSamples}.} 
  \label{fig:NHits_2Signal_noSelection_normalized}
\end{figure}



Additionally, the search for disappearing tracks which targets models with charginos decaying inside the tracker did not make use of the high energy deposition of heavy particles. 
Although this variable was indeed used in the search for long-lived charged particles, this search was not optimised for intermediate lifetimes (\eg no explicit muon veto on the selected tracks was required). 
Thus, it shows less sensitivity compared to the disappearing track search in the lifetime region between $35\,\text{cm} \lesssim c\tau \lesssim 100\,\text{cm}$ (see Fig.~\ref{fig:pMSSMplot}).\\

To conclude, the general search strategy of the here presented analysis is to unite the strategies of \cite{bib:CMS:HSCP_8TeV} and \cite{bib:CMS:DT_8TeV} and to lower the strong selection on the number of hits in these analyses in order to get an optimised selection for lifetimes around $10\,\text{cm} \lesssim c\tau \lesssim  40\,\text{cm}$.

%%%%%%%%%%%%%%%%%%%%%%%%%%%%%%%%%%%%%%%%%%%%%%%%%%%%%%%%%%%%%%%%%%%%%%%%%%%%%%%%%%%%%%%%%%%%%%%%%%%%%%%%%%%%%%%%%%%%%%%%%%%%%%%%%%%%%%%%%%%%%%%%%%%%%%%%%%%%%%%%%%%%%%%%%%%%%%%%%%%%
%%%%%%%%%%%%%%%%%%%%%%%%%%%%%%%%%%%%%%%%%%%%%%%%%%%%%%%%%%%%%%%%%%%%%%%%%%%%%%%%%%%%%%%%%%%%%%%%%%%%%%%%%%%%%%%%%%%%%%%%%%%%%%%%%%%%%%%%%%%%%%%%%%%%%%%%%%%%%%%%%%%%%%%%%%%%%%%%%%%%
